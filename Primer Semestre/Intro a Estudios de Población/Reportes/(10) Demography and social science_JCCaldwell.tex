\documentclass[11pt,spanish,letterpaper]{article}
\usepackage[spanish,activeacute]{babel}
\usepackage{caption}
\usepackage{subcaption}
\usepackage[latin1]{inputenc}
\usepackage{babel}
\usepackage{picture}
\usepackage{url}
\usepackage{mathrsfs}
\usepackage{amssymb,amsthm,amsmath,latexsym}
\usepackage[round]{natbib}
\usepackage{fancyhdr}
\theoremstyle{plain}
\newtheorem{teo}{Teorema}
\newtheorem{prop}[teo]{Proposicion}
\newtheorem{defi}[teo]{Definicion}
\newtheorem{obs}[teo]{Observacion}
\newtheorem{lem}[teo]{Lema}
\newtheorem{cor}[teo]{Corolario}
\usepackage[pdftex]{color,graphicx}
\newcommand{\sgn}{\mathop{\mathrm{sgn}}}
%\author{Adriana Perez-Arciniega Soberon} 
\begin{document}
\title{Demography and Social Science}
\date{}
\maketitle
El art\'iculo de \cite{caldwell1996demography} trata de explicar qu\'e es la demograf\'ia y c\'omo esta se relaciona con las ciencias sociales. A lo largo de la historia y de distintos textos, la definici\'on de demograf\'ia ha cambiado. Particularmente, se enfoca en la revista \textit{Population Studies} y en el trayecto de la misma definiend el campo de estudio.\\%En la \textit{Encyclopaedia of the Social Sciences} publicada en 1930, el autor A.B. Wolfe concluye que la demograf\'ia es m\'as bien considerada como una disciplina que considerando el colectivo, es la que lleva el inventario y el an\'alisis de la poblaci\'on y sus procesos vitales. Este es uno de los primeros acercamientos a definir puntualmente la demograf\'ia, a\'un tiempo despu\'es,  en 1968 en la \textit{International Encyclopedia of the Social Sciences} en 1968 se sigue considerando a la demograf\'ia como el estudio cu\'antico de los procesos vitales, aunque ahora se le a$\tilde{n}$ade la noci\'on de la utilizaci\'on de estad\'isticas vitales y censos o el \'enfasis en el enfoque de ser un estudio de poblaciones, basado en la colectividad.
\\
En un principio, la demograf\'ia era considerada como una sub disciplina que sirve a las dem\'as ciencias sociales pero conforme fueron sucediendo diversos sucesos que impactaron la estructura de la poblaci\'on, como el baby-boom a mediados del siglo pasado, flujos migratorios, transici\'on demogr\'afica, cambios en la estructuras familiares y la historia de la poblaci\'on; se ha concluido que la demograf\'ia y el dem\'ografo siguen interesados en explicar los comportamientos y el cambio en fen\'omenos sociales. Esto los separa de las dem\'as ciencias y cient\'ificos sociales.\\
\\
\cite{caldwell1996demography} postula que la diferencia fundamental entre los dem\'ografos y cualquier otro cient\'ifico social es el enfoque con el cual el primero aborda un problema de estudio. Esto es que entre las caracter\'isticas del dem\\ografo est\'a no conformarse con una teor\'ia hasta que esta explique satisfactoriamente la realidad y, en contraparte, no conf\'ian en ninguna realidad que no pueda ser cuantificable; es decir, que el dem\'ografo tiene una metodolog\'ia espec\'ifica que busca tener modelos cuantitativos que describan cualquier fen\'omeno social.\\
\\
Este an\'alisis al campo demogr\'afico es interesante al poner m\'as peso en la actitud de las personas que hacen el campo que en el campo mismo para definirlo. Es decir, que lo que define a la demograf\'ia es la curiosidad incansable del dem\'ografo, su adaptabilidad a nuevos temas, el rigor metodol\'ogico y el intere\'es por teor\'ias que reflejen la realidad. Adem\'as, demostrando dichas habilidades a lo largo de la trayectoria de la revista \textit{Population Studies}, dando importancia a distintos temas a lo largo del tiempo conforme fueron cambiando las inquietudes, pero siempre con las mismas caracter\'isticas de metodolog\'ia. As\'i, la demograf\'ia puede acercarse por momentos a ser una ciencia social, alejarse por otros, servirse de ellas o servirles; pero siempre en una estrecha relaci\'on. 
\bibliographystyle{apalike}
\bibliography{bib_COLMEX}
\end{document}
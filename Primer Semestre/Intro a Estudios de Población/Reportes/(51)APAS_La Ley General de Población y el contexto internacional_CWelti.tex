\documentclass[11pt,spanish,letterpaper]{article}
\usepackage[spanish,activeacute]{babel}
\usepackage{caption}
\usepackage{subcaption}
%\usepackage[latin1]{inputenc}
\usepackage{babel}
\usepackage{picture}
\usepackage{url}
\usepackage{mathrsfs}
\usepackage{amssymb,amsthm,amsmath,latexsym}
\usepackage[round]{natbib}
\usepackage{fancyhdr}
\theoremstyle{plain}
\newtheorem{teo}{Teorema}
\newtheorem{prop}[teo]{Proposicion}
\newtheorem{defi}[teo]{Definicion}
\newtheorem{obs}[teo]{Observacion}
\newtheorem{lem}[teo]{Lema}
\newtheorem{cor}[teo]{Corolario}
\usepackage[pdftex]{color,graphicx}
\newcommand{\sgn}{\mathop{\mathrm{sgn}}}
\setlength{\textwidth}{12.6cm}
\setlength{\textheight}{19cm}
\begin{document}
\begin{flushleft}
Adriana P\'erez-Arciniega Sober\'on
\end{flushleft}
\begin{center}
\textbf{La Ley General de Poblaci\'on y el contexto internacional}
\end{center}
\cite{welti2005ley} analiza la participaci\'on de los diversos actores en el dise$\tilde{n}$o de la Ley General de Poblaci\'on. En primer lugar, se establece que aunque las tasas de natalidad han disminuido, la poblaci\'on sigue aumentando con sus propias necesidades que el estado deber\'ia de poder cubrir; adem\'as do de los procesos de envejecimiento hacen necesarias pol\'iticas poblacionales de acuerdo a este nuevo panorama.\\
\\
Se establece que uno de los mayores factores externos que influyeron en la formulaci\'on de la nueva ley poblacional, fue la presi\'on estadounidense que consideraba el crecimiento poblacional mexicano como una amenaza a su seguridad. Dado que cuando el crecimiento poblacional de una naci\'on se vuelve amenaza de seguridad nacional para otra, los esfuerzos de controlar la poblaci\'on se vuelven globales. Otro de los factores externos que influyeron fue la reconstrucci\'on de los pa\'ises despu\'es de la Segunda Guerra Mundial, dado que se cre\'ia que era perjudicial para la econom\'ia una tasa de fecundidad alta pues habr\'ia menos recursos para repartir.\\
\\
Para la d\'ecada de los setenta, se crearon varias agencias internacionales dedicadas a las investigaciones demogr\'aficas. Con estas agencias, financiadas en su mayor\'ia por Estados Unidos, se puso la baja en las tasas de fecundidad como uno de los principales objetivos ayudado por el descubrimiento de la pastilla anticonceptiva. Se justificaba mediante el razonamiento de que el crecimiento de la poblaci\'on era un obst\'aculo central para el desarrollo. Uno de los mayores contratiempos que tuvo que sortear las iniciativas estadounidenses de control de natalidad en pa\'ises como M\'exico fue la pronunciaci\'on de la Iglesia Cat\'olica respecto a la planificaci\'on familiar, cambiando de ser un atentado a las leyes naturales de Dios o aceptar la planificaci\'on natural aunque continuando su repudio a m\'etodos de interrupci\'on de una concepci\'on ya en curso. \\
\\
En conclusi\'on, el autor propone que no se debe bajar la guardia en respecto a la planeaci\'on de la evoluci\'on demogr\'afica en un pa\'is como M\'exico. Deben seguirse considerando como prioridades los efectos del llamado 'bono demogr\'afico' y c\'omo aprovecharlos de la mejor manera para evitar que estos se vuelvan en una 'trampa demogr\'afica'.
\bibliographystyle{apalike}
\bibliography{bib_COLMEX}
\end{document}
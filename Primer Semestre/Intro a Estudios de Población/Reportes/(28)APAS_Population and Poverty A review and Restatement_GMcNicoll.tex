\documentclass[11pt,spanish,letterpaper]{article}
\usepackage[spanish,activeacute]{babel}
\usepackage{caption}
\usepackage{subcaption}
%\usepackage[latin1]{inputenc}
\usepackage{babel}
\usepackage{picture}
\usepackage{url}
\usepackage{mathrsfs}
\usepackage{amssymb,amsthm,amsmath,latexsym}
\usepackage[round]{natbib}
\usepackage{fancyhdr}
\theoremstyle{plain}
\newtheorem{teo}{Teorema}
\newtheorem{prop}[teo]{Proposicion}
\newtheorem{defi}[teo]{Definicion}
\newtheorem{obs}[teo]{Observacion}
\newtheorem{lem}[teo]{Lema}
\newtheorem{cor}[teo]{Corolario}
\usepackage[pdftex]{color,graphicx}
\newcommand{\sgn}{\mathop{\mathrm{sgn}}}
\setlength{\textwidth}{12.6cm}
\setlength{\textheight}{19cm}
\begin{document}
\begin{flushleft}
Adriana P\'erez-Arciniega Sober\'on
\end{flushleft}
\begin{center}
\textbf{Population and Poverty: A review and Restatement}
\end{center}
En el libro de \cite{mcnicoll1997population} plantea que para 1990 se estimaba que exist\'ian mas de mil millones de personas con menos de un d\'olar diario para sobrevivir, consider\'andose pobreza absoluta. Aunque se pueden tener teor\'ias acerca de qu\'e provoca la pobreza, como el r\'apido crecimiento poblacional, en realidad, hay v\'inculos y relaciones indirectas que propician la pobreza.\\
\\
La relaci\'on entre el crecimiento poblacional y la pobreza no se ha esclarecido del todo, debido a que una primera teor\'ia indicar\'ia que entre mayor poblaci\'on haya, mayor ser\'a la fuerza de trabajo y de consumo, conllevando un crecimiento econ\'omico. La realidad resulta ser m\'as complicada que eso. Al considerarse la pobreza como un problema de falta de ingreso meramente econ\'omico, puede medirse a trav\'es de m\'etodos cuantitativos pero ignorar\'ia las dimensiones intangibles de la pobreza. Para poder medir estas dimensiones se tendr\'ian que desarrollar m\'etodos antropol\'ogicos.\\
\\
En una primera categor\'ia, se define la pobreza extrema como la situaci\'on donde una poblaci\'on se caracteriza por malnutrici\'on, analfabetismo y enfermedades contagiosas; aunque en t\'erminos econ\'omicos, se define como la poblaci\'on incapaz de alcanzar el b\'asico nivel de consumo dado el ingreso per c\'apita. La brecha de pobreza se define como el d\'eficit agregado de ingresos de todos los que est\'an por debajo de la l\'inea de pobreza como una proporci\'on del consumo agregado. Sin embargo, se pueden plantear medidad alternaticas como indicadores de salud, tasas de mortalidad infantil, indicadores de calidad de vida, etc. Tambi\'en, existe la noci\'on de que la pobreza se 'transmite' de generaci\'on en  generaci\'on, mediante las restricciones educativas o alimentarias en la infancia.\\
\\
El problema con la primera visi\'on econ\'omica del crecimiento poblacional era que, mediante teor\'ias m\'as complejas, se concluy\'o que a un nivel macro, a mayor poblaci\'on, se requiere una mayor inversi\'on para sostener los m\'inimos est\'andares, lo cual reduce el capital individual para consumo. Con los problemas de la econom\'ia formal de absorber el n\'umero de trabajadores existentes, comienzan a desarrollarse la econom\'ia informal y la disminuci\'on de la poblaci\'on y la econom\'ia rural. Uno de los mayores argumentos que vincula el crecimiento poblacional con la pobreza, es el desorden social que dicho crecimiento puede provocar. Dado que este desorden social paralizar\'ia cualquier actividad econ\'omica.\\
\\
A un nivel micro, se puede considerar que el crecimiento poblacional y sus altas tasas de fecundidad impiden a las familias invertir en la educaci\'on y condiciones necesarias para sus hijos 'transmitiendo la pobreza'. Otro de los efectos a las famlias es que debido al crecimiento poblacional, los recursos naturales se ven agotados, impidiendo a nuevas generaciones cubrir sus necesidades b\'asicas de la manera adecuada. As\'i como se considera que la pobreza se transmite entre generaciones, tambi\'en se podr\'ia argumentar que se transmite a los sectores ya de por s\'i muy vulnerados de la sociedad, como minor\'ias \'etnicas.\\
\\
Al igual, se puede hablar de c\'omo afecta la pobreza a los distintos grupos sociodemogr\'aficos. Por ejemplo, a los adultos maypres y discapacitados pueden considerarse destinados a la pobreza, dado que no existe la misma base social necesaria de poblaci\'on econ\'omicamente activa para sostenerlos. Es el mismo caso de los refugiados, pol\'iticos o ambientales, que carecen de bienes y, muchas veces, tienen concepciones culturales diferentes relacionadas a la fecundidad.\\
\\
Con el cambio de enfoque derivado de El Cairo en el dise$\tilde{n}$o de pol\'iticas de salud reproudctiva, se deben tomar en cuenta el enfoque de pobreza y crecimiento poblacional. Es por eso que, al plantearse construir una pol\'itica de alivio a la pobreza, no se puede solo enfocar en los grupos marginados o sectores que son vulnerables; sino tratar de hacer la pol\'itica lo m\'as amplia posible para que beneficie a la mayor cantidad de gente. Aunque se debe intentar por un alivio de la pobreza integral, no solo ocuparse del crecimiento econ\'omico, sino atacar todas las dimensiones del problema.\\
\\
En el libro se trata de establecer conclusiones sobre c\'omo opera la pobreza en las poblaciones. Se trata de identificar las relaciones entre crecimiento poblacional, distribuci\'on econ\'omica y pobreza, estas relaciones pueden ser examinadas desde un punto de vista gubernamental o institucional y dise$\tilde{n}$ar pol\'iticas o intervenciones directas al respecto. Estas acciones gubernamentales deben siempre tomar en cuenta que la pobreza es un problema con muchas aristas y debe resolverse de la manera m\'as integral posible.
\bibliographystyle{apalike}
\bibliography{bib_COLMEX}
\end{document}
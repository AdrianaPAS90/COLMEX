\documentclass[11pt,spanish,letterpaper]{article}
\usepackage[spanish,activeacute]{babel}
\usepackage{caption}
\usepackage{subcaption}
%\usepackage[latin1]{inputenc}
\usepackage{babel}
\usepackage{picture}
\usepackage{url}
\usepackage{mathrsfs}
\usepackage{amssymb,amsthm,amsmath,latexsym}
\usepackage[round]{natbib}
\usepackage{fancyhdr}
\theoremstyle{plain}
\newtheorem{teo}{Teorema}
\newtheorem{prop}[teo]{Proposicion}
\newtheorem{defi}[teo]{Definicion}
\newtheorem{obs}[teo]{Observacion}
\newtheorem{lem}[teo]{Lema}
\newtheorem{cor}[teo]{Corolario}
\usepackage[pdftex]{color,graphicx}
\newcommand{\sgn}{\mathop{\mathrm{sgn}}}
\setlength{\textwidth}{12.6cm}
\setlength{\textheight}{19cm}
\begin{document}
\begin{flushleft}
Adriana P\'erez-Arciniega Sober\'on
\end{flushleft}
\begin{center}
\textbf{Population and poverty in households: a review of reviews}
\end{center}
En el art\'iculo, \cite{merrick2001population} expone la relaci\'on entre la pobreza y la conformaci\'on de hogares; esto con el prop\'osito de ayudar a la mejor construcci\'on de pol\'iticas p\'ublicas que alivien la pobreza y que superen las externalidades negativas inherentes al mercado. Adem\'as de discutir c\'omo la conformaci\'on de hogares y la fecundidad en estos pueden vulnerar algunos hogares a la misma.\\
\\
Se hace referencias a estudios que indican que la alta fecundidad en los hogares pobres no es una decisi\'on econ\'omica irracional, dado que en el corto plazo, una mayor cantidad de hijos significaba una mayor capital de producci\'on y de apoyo para cuando los padres llegaran a la vejez. Mientras que otros autores exponen que el tama$\tilde{n}$o del hogar y la pobreza tienen una relaci\'on lineal, aunque no se concluye la relaci\'on de causalidad en \'esta. Adem\'as, de definir que dimensiones de la pobreza se pueden incluir en el an\'alisis, dado que no solo se habla de pobreza econ\'omica sino cultural, salud o educaci\'on y c\'omo estas din\'amicas afectan a los distintos integrantes del hogar, normalmente se observa que son las mujeres quienes llevan m\'as peso de la pobreza.\\
\\
En la d\'ecada de los 80's, se encontr\'o que existe una relaci\'on negativa entre el gasto per c\'apita y el tama$\tilde{n}$o del hogar. Es m\'as, no solo influye el tama$\tilde{n}$o del hogar, sino tambi\'en el orden de nacimiento de los hijos y su espaciamiento para poder determinar que en el dise$\tilde{n}$o de pol\'iticas p\'ublicas, deben tomarse todos estos elementos en consideraci\'on. De manera m\'as reciente, se han estudiado tambi\'en los efectos que tienen factores externos en la pobreza en los hogares. Por ejemplo, como las mujeres que son jefas de hogar se enfrentan a doble jornada como proveedor econ\'omico y de cuidados, la asignaci\'on de recursos por parte de los padres favoreciendo a los hijos varones e, incluso, los efectos de los programas gubernamentales y la trasici\'on demogr\'afica en la pobreza de los hogares.\\
\\
\cite{merrick2001population} concluye diciendo que mientras los dem\'ografos permanezcan obstinados en encontrar relaciones negativas entre el tama$\tilde{n}$o de la familia y la riqueza en el hogar, ser\'a muy dif\'icil llegar a conclusiones certeras. No se debe olvidar la dificultad en establecer relaciones causales y que las decisiones de las familias sobre la fecundidad, normalmente son racionales. Se debe tomar un enfoque mucho m\'as amplio que tome en cuenta factores internos y externos del hogar para poder implementar las mejores pol\'iticas p\'ublicas de alivio de la pobreza.
\bibliographystyle{apalike}
\bibliography{bib_COLMEX}
\end{document}
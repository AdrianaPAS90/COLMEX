\documentclass[11pt,spanish,letterpaper]{article}
\usepackage[spanish,activeacute]{babel}
\usepackage{caption}
\usepackage{subcaption}
%\usepackage[latin1]{inputenc}
\usepackage{babel}
\usepackage{picture}
\usepackage{url}
\usepackage{mathrsfs}
\usepackage{amssymb,amsthm,amsmath,latexsym}
\usepackage[round]{natbib}
\usepackage{fancyhdr}
\theoremstyle{plain}
\newtheorem{teo}{Teorema}
\newtheorem{prop}[teo]{Proposicion}
\newtheorem{defi}[teo]{Definicion}
\newtheorem{obs}[teo]{Observacion}
\newtheorem{lem}[teo]{Lema}
\newtheorem{cor}[teo]{Corolario}
\usepackage[pdftex]{color,graphicx}
\newcommand{\sgn}{\mathop{\mathrm{sgn}}}
\setlength{\textwidth}{12.6cm}
\setlength{\textheight}{19cm}
\begin{document}
\begin{flushleft}
Adriana P\'erez-Arciniega Sober\'on
\end{flushleft}
\begin{center}
\textbf{Respuestas de pol\'itica a los problemas de pobreza y desigualdad en el mundo en desarrollo}
\end{center}
\cite{berry2003respuestas} sostiene que la \'unica manera de medir la eficacia de los m\'etodos utilizados para aliviar la pobreza, depende de la definici\'on del t\'ermino mismo. Si se habla de pobreza absoluta, entonces se tiene una discusi\'on concerniente al crecimiento en general, mientras que si se habla de pobreza relativa, entonces se est\'a hablando de desigualdad. Por ejemplo, de acuerdo a las mediciones del PIB se puede concluir una reducci\'on en la pobreza absoluta en los pa\'ises en v\'ias de desarrollo; sin embargo, este crecimiento est\'a plagado de desigualdades.\\
\\
Las pol\'iticas conservadoras, normalmente est\'an enfocadas en el crecimiento absoluto de la econom\'ia nacional como la pol\'itica macroecon\'omica conservadora en la cual se usca reducir la inflaci\'on, sin embargo, estas pol\'iticas no benefician de manera equitativa en la distribuci\'on de la pobreza. La pol\'itica de ahorro e inversi\'on pueden no ser tan eficientes si solo benefician a la clase que tiene la posibilidad financiera de procurarse el ahorro e invertirlo de manera eficaz. La acumulaci\'on de capital humano, se considera uno de los m\'as importantes, y la convicci\'on de aumentarlo con aquellos que se encuentran pr\'oximos al extremo inferior de los grados educativos y de ingresos se considera eficaz en la lucha contra la pobreza, pero no se resuelven ni se conoce m\'as sobre las brechas educativas y medios para combartirla. El cambio tecnol\'ogico, igualmente solo beneficia a aquellos sectores poblacionales que pueden acceder a la tecnolog\'ia, adem\'as de hacer obsoleto mucho del capital humano acumulado en las sociedades.\\
\\
Por otro lado, se discuten aquellas pol\'iticas que se enfocan en la distribuci\'on equitativa en el combate a la pobreza. Entre estas se encuentran las pol\'iticas de redistribuci\'on de los activos productivos, por ejemplo, mediante expropiaciones permitiendo que todo el capital humano tenga acceso a medios de producci\'on. Otra es las pol\'iticas de apoyo a granjas pequeñas y empresas micro, peque$\tilde{n}$as y medianas o pol\'iticas de alivio directo de pobreza.\\
\\
Mediante el an\'alisis de estas pol\'iticas se puede concluir que el desarrollo integral de la poblaci\'on no depende necesariamente de alcanzar un nivel de ingreso, sino en democratizar los medios educativos y de producci\'on. Por lo que las pol\'iticas m\'as eficaces son aquellas que promueven el ahorro, inversi\'on educativa en el capital humano y la mayor participaci\'on de la poblaci\'on posible. 
\bibliographystyle{apalike}
\bibliography{bib_COLMEX}
\end{document}
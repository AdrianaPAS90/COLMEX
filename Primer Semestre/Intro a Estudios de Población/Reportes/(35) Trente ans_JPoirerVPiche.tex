\documentclass[11pt,spanish,letterpaper]{article}
\usepackage[spanish,activeacute]{babel}
\usepackage{caption}
\usepackage{subcaption}
%\usepackage[latin1]{inputenc}
\usepackage{babel}
\usepackage{picture}
\usepackage{url}
\usepackage{mathrsfs}
\usepackage{amssymb,amsthm,amsmath,latexsym}
\usepackage[round]{natbib}
\usepackage{fancyhdr}
\theoremstyle{plain}
\newtheorem{teo}{Teorema}
\newtheorem{prop}[teo]{Proposicion}
\newtheorem{defi}[teo]{Definicion}
\newtheorem{obs}[teo]{Observacion}
\newtheorem{lem}[teo]{Lema}
\newtheorem{cor}[teo]{Corolario}
\usepackage[pdftex]{color,graphicx}
\newcommand{\sgn}{\mathop{\mathrm{sgn}}}
\setlength{\textwidth}{12.6cm}
\setlength{\textheight}{19cm}
\begin{document}
\begin{flushleft}
Adriana P\'erez-Arciniega Sober\'on
\end{flushleft}
\begin{center}
\textbf{Trente ans de recherche explicative en d\'emographie. R\'eflexions autour des dangers du cloisonnement}
\end{center} 
\cite{poirier1999trente} proponen que la debilidad m\'as grande que ha presentado la demograf\'ia en los \'ultimos treinta a$\tilde{n}$os ha sido la de comparmentalizar la investigaci\'on y el conocimiento, aisl\'andolo y dificultando su progreso. Esta comparmentalizaci\'on se divide en cinco elementos claves de la disciplina: los objetos de investigaci\'ion, los niveles de an\'alisis, los factores explicativos, las corrientes te\'oricas y los lugares de producci\'on del conocimiento.\\
\\ 
El aislamiento de los objetos de investigaci\'on se refiere a como un tema puede ser dividido de tal manera que no hay comunicaci\'on, por ejemplo, la migraci\'on que se divide entre nacional e internacional. En el aislamiento de los niveles de an\'alisis se habla de que el estudo y an\'alisis ha ido de una visi\'on macro a una micro, mientras que se deber\'ia desarrollar un an\'alisis multinivel que concilie las teor\'ias macro con el acercamiento emp\'irico micro. Lo que se refiere a el aislamiento de los factores explicativos es que ha habido varias corrientes dedicadas a estudiar los factores explicativos de fen\'omenos demogr\'aficos como la fecundidad; sin embargo, quienes estudian estos factores no hablan entre s\'i dejando an\'alisis incompletos.\\
\\
En lo que se refiere al aislamiento de las corrientes te\'oricas y lugares de producci\'on del conocimiento, el art\'iculo solo se enfoca en dos escuelas prominentes: la escuela estadounidense y la francesa y comenta que por el aislamiento en corrientes te\'oricas, la escuela estadounidense sigue desarrollando teor\'ias que la francesa descart\'o hace d\'ecadas aun cuando cuenta con mayor prestigio. Tambi\'en se mencionan las escuelas latinoamericana y la africana, que en el esquema global quedan totalmente descartadas. La escuela estadounidense pareciera que se preocupa m\'as con la fecundidad de pa\'ises del tercer mundo; sin embargo, hicieron grandes aportes a la integraci\'on de factore biom\'edicos y socioecon\'omicos a la discusi\'on de mortalidad. En la escuela francesa sigue muy activo el debate entre demograf\'ia formal y demograf\'ia social, pero la tendencia se inclina por el an\'alisis demogr\'afico con una metodolog\'ia m\'as r\'igida y cr\'tica de los procesos demogr\'aficos. Mientras que las escuelas de la periferia se han apegado a las escuelas del centro (la latinoamericana a la estadounidense y la africana a la francesa), estas escuelas perif\'ericas han hecho grandes aportes sobre la demograf\'ia hist\'orica, la migraci\'on y estrategias de supervivencia.\\
\\
El art\'iculo insta a la comunidad demogr\'afica a acercarse los unos a los otros con un enfoque cualitativo sobre los trabajos en \'animo de juntar el conocimiento y llegar a verdaderos progresos en la disciplina. 
\bibliographystyle{apalike}
\bibliography{bib_COLMEX}
\end{document}
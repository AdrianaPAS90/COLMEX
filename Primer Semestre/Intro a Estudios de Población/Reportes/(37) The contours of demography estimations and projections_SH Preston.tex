\documentclass[11pt,spanish,letterpaper]{article}
\usepackage[spanish,activeacute]{babel}
\usepackage{caption}
\usepackage{subcaption}
\usepackage[latin1]{inputenc}
\usepackage{babel}
\usepackage{picture}
\usepackage{url}
\usepackage{mathrsfs}
\usepackage{amssymb,amsthm,amsmath,latexsym}
\usepackage[round]{natbib}
\usepackage{fancyhdr}
\theoremstyle{plain}
\newtheorem{teo}{Teorema}
\newtheorem{prop}[teo]{Proposicion}
\newtheorem{defi}[teo]{Definicion}
\newtheorem{obs}[teo]{Observacion}
\newtheorem{lem}[teo]{Lema}
\newtheorem{cor}[teo]{Corolario}
\usepackage[pdftex]{color,graphicx}
\newcommand{\sgn}{\mathop{\mathrm{sgn}}}
%\author{Adriana Perez-Arciniega Soberon} 
\begin{document}
\title{The Contours of Demography: Estimates and Projections}
\date{}
\maketitle
El art\'iculo de \cite{preston1993contours}, al igual que \cite{caldwell1996demography} empiezan hablando sobre la dificultad en definir con precisi\'on la demograf\'ia. Igualmente, ambos art\'iculos abordan que la principal diferencia entre los dem\'ografos y otros cient\'ificos sociales es los m\'etodos y el enfoque con el cual se abordan las preguntas que se quieren contestar, utilizando la tecnolog\'ia disponible.\\
\\
\cite{preston1993contours} alega que la demograf\'ia, a diferencia de las dem\'as ciencias sociales, es muy sensible a las demandas externas; es decir, que las l\'ineas de investigaci\'on que se persiguen derivan de problemas mundiales o inquietudes sociales m\'as de que de la misma estructura interna de la ciencia y que justo por estas razones, los avances en el campo han sido menos frecuentes que en otros, como pudieran ser la econom\'ia o la estad\'istica. Sin embargo, debido a la misma unicidad del modelo donde se conjuntan habilidades cuantitavivas con alta capacidad de an\'alisis de los contextos sociales, han logrando investigaciones innovadoras y que, incluso han cambiado los paradigmas existentes, en algunos temas como fertilidad y mortalidad. Utilizando la tecnoclog\'ia del momento para hacer cada vez an\'alisis m\'as fino de acuerdo al incremento en facilidad y disponibilidad de datos cada vez m\'as granulados y bases de datos m\'as grandes.\\
\\
El art\'iculo presenta algunas l\'ineas de investigaci\'on en las cuales la demograf\'ia se est\'a enfocando. Entre ellos est\'an la equidad de los recursos, la conexi\'on entre el crecimiento poblacional y el cambio ambiental, los impactos de la migraci\'on entre zonas a las estructuras familiares, sociales y econ\'omicas;tanto en los lugares de origen de los migrantes como en los lugares de destino para inmigraci\'on. Igualmente, persisten temas importante como la fecundidad y el envejecimiento. Uno de los nuevos temas de la fecundidad es ver c\'omo los cambios en \'esta tienen impactos en la estructura familiar y en los procesos vitales de la sociedad. Al igual, con el tema del envejecimiento, la demograf\'ia tiene el reto de ir m\'as all\'a de la descripci\'on e intentar conectar las estruturas de la pir\'amide poblacional con la experiencia y ciclo de vida del individuo.\\
%El art\'iculo tambi\'en nombra l\'ineas de investigaci\'on futuras que la demograf\'ia puede abordar, de acuerdo a los procesos sociales de actualidad. Entre estos temas est\'an los temas de equidad, en el cual la demograf\'ia ha hecho un trabajo importante en describir las estructuras sociales que rodean fen\'omenos como la pobreza. Por otro lado, las condiciones actuales exigen un an\'alisis que conecte el crecimiento poblacional con el cambio ambiental; sin embargo \cite{preston1993contours} expone que este es un tema muy fuera del alcance de cualquier dem\'ografo, por m\'as intentos que se realicen. Igualmente, un tema importante es la migraci\'on debido a la disparidad de la riqueza entre zonas, lo cual provoca un cambio en las estructuras familiares, sociales y econ\'omicas; tanto en los lugares de origen de los migrantes como en los lugares de destino para inmigraci\'on.\\
\\
%Los temas perennes de estudios demogr\'aficos tienen que ver con fecundidad y envejecimiento. Los cambios en la estructura familiar y cambios de fecundidad y c\'omo se afectan los procesos vitales de la sociedad debido a estos cambios, son preguntas de inter\'es para los dem\'ografos, tratando de abordarlas con un enfoque m\'as integral que como solo una unidad econ\'omica. Al igual, con el tema del envejecimiento, la demograf\'ia tiene el reto de ir m\'as all\'a de la descripci\'on e intentar conectar las estruturas de la pir\'amide poblacional con la experiencia y ciclo de vida del individuo. 
Independientemente del tema que se estudie, son las t\'ecnicas y el enfoque del dem\'ografo los que los permiten realizar verdaderas contribuciones a las ciencias sociales; adem\'as de servir como punto de partida para varios estudios m\'as detallados. 
\bibliographystyle{apalike}
\bibliography{bib_COLMEX}
\end{document}
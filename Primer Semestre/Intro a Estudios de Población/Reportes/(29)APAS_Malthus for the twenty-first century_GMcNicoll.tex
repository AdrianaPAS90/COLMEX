 \documentclass[11pt,spanish,letterpaper]{article}
\usepackage[spanish,activeacute]{babel}
\usepackage{caption}
\usepackage{subcaption}
%\usepackage[latin1]{inputenc}
\usepackage{babel}
\usepackage{picture}
\usepackage{url}
\usepackage{mathrsfs}
\usepackage{amssymb,amsthm,amsmath,latexsym}
\usepackage[round]{natbib}
\usepackage{fancyhdr}
\theoremstyle{plain}
\newtheorem{teo}{Teorema}
\newtheorem{prop}[teo]{Proposicion}
\newtheorem{defi}[teo]{Definicion}
\newtheorem{obs}[teo]{Observacion}
\newtheorem{lem}[teo]{Lema}
\newtheorem{cor}[teo]{Corolario}
\usepackage[pdftex]{color,graphicx}
\newcommand{\sgn}{\mathop{\mathrm{sgn}}}
\setlength{\textwidth}{12.6cm}
\setlength{\textheight}{19cm}
\begin{document}
\begin{flushleft}
Adriana P\'erez-Arciniega Sober\'on
\end{flushleft}
\begin{center}
\textbf{Malthus for the twenty-first century}
\end{center}
\cite{mcnicoll1998malthus} establece que, a pesar del aumento poblacional nada menor de 8 billones de personas para 2050, las demandas de Malthus han sido atendidas. Una de los m\'as grandes factores, ha sido los desarrollos tecnol\'ogicos masivos en la industria y en el campo, por lo que la discusi\'on sobre pobreza y desarrollo ha sobrepasado la restricci\'on de recursos b\'asicos. Sin embargo, esto no implica que la totalidad del trabajo de Malthus sea irrelevante en la actualidad, espec\'ificamente en los temas de Estado y sociedad, distribuci\'on y naturaleza, a\'un hay mucho que rescatar.\\
\\
Uno de los resultados de los trabajos de Malthus es que la discusi\'on sobre alimentos y poblaci\'on, a trav\'es del tiempo, se convirti\'o en una discusi\'on sobre como el orden social y el gobierno influenc\'ian los resultados demogr\'aficos y econ\'omicos. Podr\'ia argumentarse que Malthus abogaba por un gobierno libertario y un Estado minimalista que diera lugar a una sociedad que permitiera la acumulaci\'on de capital en la prosperidad y en la miseria, un gobierno que persiguiera la disminuci\'on de la poblaci\'on.\\
\\
En el tema de la distribuci\'on de recursos bajo ideas malthusianas, empieza en la definici\'on de derechos y ejecuci\'on. La declaraci\'on de un derecho, aunque satisfactorio, no tiene consecuencias pr\'acticas inmediatas por lo que la proliferaci\'on de declarac\'on de derechos humanos no se alinea con los objetivos. Igualmente con la ejecuci\'on de m\'ultiples programas, nacionales e internacionales, para el alivio de la pobreza, subsidios a las necesidades b\'asicas, etc; no concuerdan con las ideas malthusianas de que estos programas a la larga traen muchos m\'as efectos negativos que los inmediatos positivos.\\
\\
En lo que respecta a la naturaleza, Malthus la ve\'ia como otro territorio a ser conquistado por el hombre. Por lo que, cualquier alusi\'on a deterioro ambiental podr\'ia ser interpretado como un efecto secundario al desarrollo econ\'omico o, incluso, producto de corrupci\'on gubernamental. Sin embargo, los retos que presenta el cambio clim\'atico actualmente, rebasan por mucho cualquier predicci\'on realizada por Malthus. Aunadas a nuevas crisis pol\'iticas y a la compleja din\'amica internacional, se pueden estar creando nuevas situaciones que no pueden ser estudiadas bajo una \'optica estrictamente malthusiana.\\
\\
Frente al crecimiento poblacional actual, focalizado principalmente en pa\'ises en v\'ias de desarrollo, los est\'andares m\'inimos de vivienda y con atenci\'on a las consecuencias ambientales; es necesario crear soluciones s\'olidas sin la ayuda de visiones ut\'opicas, por lo que Malthus puede ser un buen principio a estudiar.
\bibliographystyle{apalike}
\bibliography{bib_COLMEX}
\end{document}
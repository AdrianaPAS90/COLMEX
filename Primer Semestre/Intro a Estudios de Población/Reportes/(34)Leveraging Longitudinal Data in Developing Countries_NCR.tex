\documentclass[10pt,spanish,letterpaper]{article}
\usepackage[spanish,activeacute]{babel}
\usepackage{caption}
\usepackage{subcaption}
%\usepackage[latin1]{inputenc}
\usepackage{babel}
\usepackage{picture}
\usepackage{url}
\usepackage{mathrsfs}
\usepackage{amssymb,amsthm,amsmath,latexsym}
\usepackage[round]{natbib}
\usepackage{fancyhdr}
\theoremstyle{plain}
\newtheorem{teo}{Teorema}
\newtheorem{prop}[teo]{Proposicion}
\newtheorem{defi}[teo]{Definicion}
\newtheorem{obs}[teo]{Observacion}
\newtheorem{lem}[teo]{Lema}
\newtheorem{cor}[teo]{Corolario}
\usepackage[pdftex]{color,graphicx}
\newcommand{\sgn}{\mathop{\mathrm{sgn}}}
\setlength{\textwidth}{12.6cm}
\setlength{\textheight}{19cm}
\begin{document}
\begin{flushleft}
Adriana P\'erez-Arciniega Sober\'on
\end{flushleft}
\begin{center}
\textbf{Leveraging Longitudinal Data in Developing Countries}
\end{center}
El Comit\'e de Poblaci\'on Nacional (\cite{national2002leveraging}) realiz\'o un taller en 2001 donde se discutieron los m\'etodos longitudinales en la investigaci\'on demogr\'afica y de salud en pa\'ises en v\'ias de desarrollo. Se entienden como m\'etodos longitudinales a los estudios en los cuales una poblaci\'on es estudiada/entrevistada por un largo per\'iodo de tiempo. En el taller se distinguier\'on tres clases de estudios longitudinales: Estudios panel, en los cuales la muestra de estudio y los temas de investigaci\'on son m\'as amplias y toman como unidad al hogar. Los estudios de cohorte son un subconjunto de los estudios panel y se sigue una muestra de poblaci\'on seleccionada en base a que comparten edad o caracter\'isticas en el tiempo. Finalmente, los estudios longitudinales comunitarios tambi\'en conocidos como laboratorios de poblaci\'on, en este estudio se recolecta informaci\'on de manera sistem\'atica de todos los individuos en una comunidad demarcada geogr\'aficamente. \\
\\
Los estudios longitudinales contribuyen a entender las relaciones causales al obtener informaci\'on m\'as precisa y detallada sobre los eventos en el tiempo y tienen la ventaja de que los datos longitudinales entrenan al investigador en habilidades para mirar el cambio en un individuo, mientras que se resta las caracter\'isticas del individuo que permanecen. La existencia de datos longitudinales permite a los investigadores entender mejor los procesos de desarrollo humano, social y econ\'omico y probar teor\'ias de comportamiento social para refinar el conocimiento de relaciones causales. Aunque uno de los retos es desarrollar maneras para sobreponer la corta vida de la intervenci\'on en las comunidades. Otra de las grandes contribuciones de los estudios longitudinales est\'a relacionada al proceso din\'amico de la investigaci\'on mediante el dise$\tilde{n}$o pr\'acticas de investigaci\'on muy cuidadosas, que involucran integrar, a largo plazo, el compromiso de la comunidad y el desarrollo de la infraestructura de investigaci\'on.\\
\\
Es importante mencionar que los tres estudios longitudinales descritos no son iguales en sus objetivos o en su habilidad de contestar determinadas preguntas. Se debe escoger el estudio dependiendo de la pregunta de investigaci\'on, el contexto, la representatividad, el tama$\tilde{n}$o de muestra, cobertura geogr\'afica, etc. Se compararon las fortalezas y debilidades de estos estudios en tres categor\'ias: Medici\'on, que se refiere a la poblaci\'on estudiada y se concluye que los estudios de panel cubren una poblaci\'on m\'as amplia mientras que los estudios de cohorte y longitudinales comunitarios se enfocan en subconjuntos espec\'ificos, lo cual puede afectar la generalizaci\'on de los resultados. La evaluaci\'on del programa, esta es especialmente sensible a cambios en el estudio por lo que continuidad en el mismo es de vital importancia; en este sentido, el estudio longitudinal comunitario tiene mejores resultados porque sigue a todos los indivdiuos de la comunidad durante un per\'iodo de tiempo. La \'ultima categor\'ia es an\'alisis estructural, los estudios longitudinales comunitarios no tratan de avanzar inmediantamente el conocimiento pero mejora la participaci\'on comunitaria, mientras que para los estudios de panel y cohorte sigue siendo importate pero no es expl\'icito en sus objetivos.\\
\\
A pesar de los beneficios de la investigaci\'on longitudinal, tambi\'en existen retos que deben ser abordados. Uno de los principales retos es la falta de fondos, debido a que esta clase de estudios requiere mucho y largo financiamiento que los estudios transversales por su dise$\tilde{n}$o, datos que necesitan ser recolectados repetidamente y recursos que manejen estos datos. Tambi\'en se necesitan tener buenas y s\'olidas relaciones con las comunidades y los entrevistados, pues todos los estudios longitudinales dependen de la cooperaci\'on de la comunidad. Los cambios de muestra son comunes a lo largo de la duraci\'on del estudio, se a$\tilde{n}$aden sujetos a trav\'es de nacimientos e inmigraci\'on y la muestra decrece a trav\'es de muertes y emigraci\'on, por lo que se puede decidir seguir a los migrantes, adaptar el tama$\tilde{n}$o de muestra o ignorar el cambio. Igualmente, los cambios en los protocolos de investigaci\'on pueden afectar incitando cambios en los participantes del estudio o contaminar los procesos de investigaci\'on. Por \'ultimo, el reto de la \'etica en las pr\'acticas de investigaci\'on, pues los estudios longitudinales est\'an sujetos a las mismas consideraciones que los dem\'as estudios, es decir, que deben estar enfocados a las personas, al beneficio y a la justicia; esto puede tomar la forma de privacidad de los encuestados, obtener consentimiento de su parte y asegurando que los beneficios sobrepasan los costos para la comunidad.\\
\\
Debido a lo largos y costosos que son estos estudios se proponen ideas para maximizarlos. La primera propuesta es la construcci\'on de una base de encuestas robusta pero flexible y, en la misma l\'inea, la posibilidad de a$\tilde{n}$adir variables y tipos de datos si as\'i lo requiriera el estudio. La creaci\'on de redes y colaboraciones con otras ciencias y estudios para entender mejor los resultados; esto se puede lograr tambi\'en a trav\'es de compartir de manera libre los datos que resulten de los estudios. Compartir datos libremente viene a su vez, con sus consideraciones de protecci\'on a los datos personales y autor\'ia intelectual para promover el avance libre de la ciencia y la tecnolog\'ia. As\'i, mediantes estas herramientas, reforzar la capacidad de investigaci\'on en pa\'ises en v\'ias de desarrollo.\\
\\
La meta de este taller era comparar las fortalezas y las debilidades de diferentes enfoques longitudinales, espec\'ificamente los enfoques de cohorte, panel y estudios longitudinales comunitarios.  Estos enfoques var\'ian en los objetivos de estudio y la participaci\'on de la comunidad de estudio. Un tema muy evidente del taller es la importancia de usar enfoques longitudinales que mejor se ajuste a la pregunta de investigaci\'on o el objetivo del estudio. Otro enfoque importante fue la importancia de m\'ultiples enfoques de investigaci\'on para poder tener un estudio m\'as completo y si un enfoque es m\'as  d\'ebil, se puede compensar f\'acilmente utilizando un enfoque similar.  
\bibliographystyle{apalike}
\bibliography{bib_COLMEX}
\end{document}
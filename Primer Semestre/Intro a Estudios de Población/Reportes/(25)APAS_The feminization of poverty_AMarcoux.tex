\documentclass[11pt,spanish,letterpaper]{article}
\usepackage[spanish,activeacute]{babel}
\usepackage{caption}
\usepackage{subcaption}
%\usepackage[latin1]{inputenc}
\usepackage{babel}
\usepackage{picture}
\usepackage{url}
\usepackage{mathrsfs}
\usepackage{amssymb,amsthm,amsmath,latexsym}
\usepackage[round]{natbib}
\usepackage{fancyhdr}
\theoremstyle{plain}
\newtheorem{teo}{Teorema}
\newtheorem{prop}[teo]{Proposicion}
\newtheorem{defi}[teo]{Definicion}
\newtheorem{obs}[teo]{Observacion}
\newtheorem{lem}[teo]{Lema}
\newtheorem{cor}[teo]{Corolario}
\usepackage[pdftex]{color,graphicx}
\newcommand{\sgn}{\mathop{\mathrm{sgn}}}
\setlength{\textwidth}{12.6cm}
\setlength{\textheight}{19cm}
\begin{document}
\begin{flushleft}
Adriana P\'erez-Arciniega Sober\'on
\end{flushleft}
\begin{center}
\textbf{The Feminization of Poverty: Claims, Facts and Data Needs}
\end{center}
\cite{marcoux1998feminization} expone en su art\'iculo que varias organizaciones internacionales alegan que del 1.3 mil millones de pobres en el mundo, se podr\'ia afirmar que m\'as del 50\%, en pa\'ises en v\'ias de desarollo, son mujeres. Sin embargo, al evaluar esta afirmaci\'on mediante la distribuci\'on de edad de la poblaci\'on bajo el umbral de pobreza, se debe considerar las circunstancias que han aumentado el n\'umero de mujeres en esta poblaci\'on, como el exceso de mortalidad masculina y la migraci\'on.\\
\\
Normalmente se considera que la mayor\'ia de la poblaci\'on en pobreza es femenina por que la mayor\'ia de los hogares con jefatura femenina est\'an en pobreza, aunque algunos estudios realizados en Am\'erica Latina no necesariamente concuerdan con esta afirmaci\'on. Adem\'as, de que aunque muchos higares vulnerados por pobreza est\'an encabezados por mujeres esto se puede explicar por la mortalidad diferenciada o migraci\'on. Si fueran por separaci\'on habr\'ian tambi\'en los higares unipersonales encabezados por los hombres separados; al detectar la ausencia de \'estos, se concluye que los hogares encabezados por mujeres contribuyen al exceso de pobreza femenina en tanto que los hombres que no est\'an no son pobres en s\'i mismos. Aun as\'i, el sesgo de g\'enero que afirma que las mujeres y sus hogares  son en comparaci\'on m\'as pobres que los varones, est\'a sobreestimado.\\
\\
Bas\'andose en encuestas realizadas por la Divisi\'on Estad\'istica de las Naciones Unidas, se concluye que en los pa\'ises en v\'ias de desarrollo como Bangladesh, Botswana, Ghana, Guatemala, entre otros y pa\'ises desarrollados como Alemania, Canad\'a, Sueca y Estados Unidos; la proporci\'on de mujeres en el quintil m\'as bajo de ingresos era de 53.5\% y no del 70\% como se alegaba. Aunque esta proporci\'on podr\'ia aumentar en 1.5 puntos porcentuales, aun queda muy lejos del sesgo previamente establecido.\\
\\
Aunque los datos no indiquen el nivel de feminizaci\'on de la pobreza que se alegaba en las agencias internacionales, esto no quiere decir que el sesgo no exista y que no se deban tomar acciones para combatirlo, dependiendo de la regi\'on. Para estudiar m\'as a profundidad estos alegatos, es necesario ser m\'as estricto en cuestiones metodol\'ogicas y usar m\'as m\'etodos de medici\'on, que no necesariamente est\'en basados en los ingresos.
\bibliographystyle{apalike}
\bibliography{bib_COLMEX}
\end{document}
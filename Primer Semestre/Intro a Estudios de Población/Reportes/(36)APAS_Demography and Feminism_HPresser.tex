\documentclass[11pt,spanish,letterpaper]{article}
\usepackage[spanish,activeacute]{babel}
\usepackage{caption}
\usepackage{subcaption}
%\usepackage[latin1]{inputenc}
\usepackage{babel}
\usepackage{picture}
\usepackage{url}
\usepackage{mathrsfs}
\usepackage{amssymb,amsthm,amsmath,latexsym}
\usepackage[round]{natbib}
\usepackage{fancyhdr}
\theoremstyle{plain}
\newtheorem{teo}{Teorema}
\newtheorem{prop}[teo]{Proposicion}
\newtheorem{defi}[teo]{Definicion}
\newtheorem{obs}[teo]{Observacion}
\newtheorem{lem}[teo]{Lema}
\newtheorem{cor}[teo]{Corolario}
\usepackage[pdftex]{color,graphicx}
\newcommand{\sgn}{\mathop{\mathrm{sgn}}}
\setlength{\textwidth}{12.6cm}
\setlength{\textheight}{19cm}
\begin{document}
\begin{flushleft}
Adriana P\'erez-Arciniega Sober\'on
\end{flushleft}
\begin{center}
\textbf{Demography, feminism and the science-policy nexus}
\end{center} 
En su art\'iculo, \citep{presser1997demography} es muy cr\'itica sobre el campo de la demograf\'ia en la praxis, pues no solo es susceptible a la ideolog\'ia de los investigadores (como el resto de las ciencias)  sino que adem\'as son las fuentes de financiamiento las que dictan las l\'ineas de investigaci\'on a desarrollar y al ambiente pol\'itico, quienes determinan la disponibilidad de las fuentes de informaci\'on como los censos y encuestas. Esto ha sido un claro obst\'aculo en el desarrollo de investigaciones con perspectiva de g\'enero dado que amenaza el status-quo de las instituciones de financiamiento.\\
\\
Un ejemplo de esto es que en la d\'ecada de los 60's hubo una inversi\'on muy grande en encuestas a gran escala y en distintos pa\'ises, esto derivado de el cambio en perspectiva donde se empez\'o a pensar en el crecimiento poblacional como el que afectaba el desarrollo socioecon\'omico y no al rev\'es y con este enfoque se justificaron pol\'iticas intervencionistas de EE UU en temas de fecundidad de otros pa\'ises. Sin embargo, se han logrado algunas investigaciones feministas, principalmente enfocadas en embarazo adolescente y cuidados infantiles, de manera local en EE UU.\\
\\
Actualmente, se est\'an empezando a desarrollar investigaciones sobre condiciones demogr\'aficas en pa\'ises en v\'ias de desarrollo; principalmente en la fecundidad y en las estructuras de poder a nivel institucional . El enfoque feminista consisten en enfocar estas investigaciones en el bienestar de la mujer.\\
\\
Finalmente, \citep{presser1997demography} si el g\'enero se podr\'ia convertir en un tema central en la demograf\'ia, considerando que es una de las dimensiones principales de estratificaci\'on de cualquier an\'alisis demogr\'afico. Aunque una de las justificaciones m\'as obvias ser\'ia hacer un an\'alisis enfocado en salud, se deber\'ia incluir el tema de g\'enero por si mismo y por su relaci\'on a los procesos demogr\'aficos. 
\bibliographystyle{apalike}
\bibliography{bib_COLMEX}
\end{document}
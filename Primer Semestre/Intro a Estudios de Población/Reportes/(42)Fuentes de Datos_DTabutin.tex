\documentclass[10pt,spanish,letterpaper]{article}
\usepackage[spanish,activeacute]{babel}
\usepackage{caption}
\usepackage{subcaption}
%\usepackage[latin1]{inputenc}
\usepackage{babel}
\usepackage{picture}
\usepackage{url}
\usepackage{mathrsfs}
\usepackage{amssymb,amsthm,amsmath,latexsym}
\usepackage[round]{natbib}
\usepackage{fancyhdr}
\theoremstyle{plain}
\newtheorem{teo}{Teorema}
\newtheorem{prop}[teo]{Proposicion}
\newtheorem{defi}[teo]{Definicion}
\newtheorem{obs}[teo]{Observacion}
\newtheorem{lem}[teo]{Lema}
\newtheorem{cor}[teo]{Corolario}
\usepackage[pdftex]{color,graphicx}
\newcommand{\sgn}{\mathop{\mathrm{sgn}}}
\setlength{\textwidth}{12.6cm}
\setlength{\textheight}{19cm}
\begin{document}
\begin{flushleft}
Adriana P\'erez-Arciniega Sober\'on
\end{flushleft}
\begin{center}
\textbf{Fuentes de Datos}
\end{center}
En el art\'iculo \cite{tabutin1997sistemas} habla sobre los grandes sistemas de informaci\'on: censos, encuestas y los registros continuos (registro civil, registro de poblaci\'on, laboratorio de poblaci\'on). Al momento en el que un investigador delimita el tema y la pregunta de investigaci\'on tambi\'en debe decidir sobre si los datos existentes son suficientes o necesita recolectar nuevos. De este modo, el sistema de informaci\'on seleccionado debe adaptarse a las necesidades de la investigaci\'on y no al rev\'es.\'
\\
Los censos tienen mucho tiempo siendo parte de la historia desde los llamados empadronamientos, al igual que los registros parroquiales fueron los precursores del servicio civil, lo cual demuestra que siempre ha habido un inter\'es de los gobiernos en conocer el n\'umero de sus habitantes. As\'i, se pueden delimitar seis grandes sistemas de informaci\'on: Los sistemas de registro continuo, es decir, el registro civil y el registro de poblaci\'on; los censos, que describen las estructuras demogr\'aficas, sociales y econ\'omicas; las encuestas, que son m\'as espec\'ificas y diversificadas de acuerdo a su objetivo; registros admnistrativos, la fotograf\'ia \'area y teledetecci\'on y los enfoque cualitativos, que tratan de comprender los fen\'omenos demogr\'aficos en una peque$\tilde{n}$a poblaci\'on.\\
\\
Los sistemas de registro continuo se refieren al registro permanente y obligatorio de hechos civiles (nacimientos, muertes, matrimonios), as\'i como los eventos que pueden modificar la situaci\'on de una persona (divorcio, adopci\'on, separaci\'on). Este sistema de informaci\'on permite analizar particularmente la fecundidad y la mortalidad; sin embargo, por la variaci\'on en las definiciones legales   es dif\'cil la comparaci\'on e, igualmente, debido a su falta de especificidad es com\'un que se desconozcan las caracter\'isticas de los fen\'omenos demogr\'aficos.\\
\\
Los censos son el conjunto de operaciones que obtienen, agrupan, eval\'uan, analizan, publican y difunden los datos demogr\'aficos, econ\'omicos y sociales de una poblaci\'on en un momento dado. Los censos son universales (debe abarcar a todos los habitantes de un territorio determinado), conteo individual, simultaneidad (debe llevarse a cabo en poco tiempo) y periodicidad (debe llevarse a cabo en intervalos regulares); normalmente los censos est\'an a cargo de agencias gubernamentales. Los censos son la \'unica fuente de informaci\'on sobre migraci\'on interna e indispensable para la planeaci\'on de pol\'iticas p\'ublicas, uno de sus puntos d\'ebiles es su alto costo y que su periodicidad es demasiado espaciada.\\
\\
Las encuestas, por otro lado, se concibe como un instrumento privilegiado para la recolecci\'on de datos pues tiene por objeto la estimaci\'on de algunas caracter\'isticas demogr\'aficas de la poblaci\'on a partir de la observaci\'on de solo una muestra de la misma. Las ventajas de las encuestas son su menor costo, mayor calidad de informaci\'on  y que se puede obtener informaci\'on sobre variables que no se podr\'ian obtener de otra manera.\\
\\
No se deben ignorar los datos administrativos que son toda red de informaci\'on estad\'istica recabada de manera cotidiana por una administraci\'on p\'ublica o privada con un objetivo de gesti\'on o seguimiento; estos registros se pueden referir a escolares, hospitalarios o de seguridad social. De igual manera, se deben considerar los enfoques cualitativos que aunque no se enfocan en medir los fen\'omenos demogr\'aficos se enfocan en comprenderlo contextualiz\'andolo en la comunidad. Normalmente, los m\'etodos demogr\'aficos se consideran complementarios a los cuantitativos.\\
\\
Los sistemas demogr\'aficos actuales se consideran insuficientes como proveedores de informaci\'on integral de los fen\'omenos demogr\'aficos. En algunos pa\'ises, con la ayuda de la tecnolog\'ia, se ha podido expandir el cuestionario de los censos; sin embargo, la tecnolog\'ia no ha disminuido los costos agregados y nuevos sistemas de informaci\'on, como la teledetecci\'on por sat\'elite, son completamente inalcabzables para muchos pa\'ises. Otro de los incovenientes es que en el \'animo de estandarizar y facilitar la comparaci\'on entre poblaciones, se reducen las diversidades locales, se borran las diferencias, se nulifica el an\'alisis cualitativo y se empobrece el conocimiento cient\'ifico general.\\
\\
En conclusi\'on, \cite{tabutin1997sistemas} exhorta a no perderse en tratar de recolectar m\'as datos mediante m\'as encuestas o censos cada menos a$\tilde{n}$os sino en mejorar la calidad de los sistemas existentes e implementar mayor investigaci\'on metodol\'ogica en la recolecci\'on de datos.
\bibliographystyle{apalike}
\bibliography{bib_COLMEX}
\end{document}
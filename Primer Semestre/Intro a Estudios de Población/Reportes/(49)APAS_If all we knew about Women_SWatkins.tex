\documentclass[11pt,spanish,letterpaper]{article}
\usepackage[spanish,activeacute]{babel}
\usepackage{caption}
\usepackage{subcaption}
%\usepackage[latin1]{inputenc}
\usepackage{babel}
\usepackage{picture}
\usepackage{url}
\usepackage{mathrsfs}
\usepackage{amssymb,amsthm,amsmath,latexsym}
\usepackage[round]{natbib}
\usepackage{fancyhdr}
\theoremstyle{plain}
\newtheorem{teo}{Teorema}
\newtheorem{prop}[teo]{Proposicion}
\newtheorem{defi}[teo]{Definicion}
\newtheorem{obs}[teo]{Observacion}
\newtheorem{lem}[teo]{Lema}
\newtheorem{cor}[teo]{Corolario}
\usepackage[pdftex]{color,graphicx}
\newcommand{\sgn}{\mathop{\mathrm{sgn}}}
\setlength{\textwidth}{12.6cm}
\setlength{\textheight}{19cm}
\begin{document}
\begin{flushleft}
Adriana P\'erez-Arciniega Sober\'on
\end{flushleft}
\begin{center}
\textbf{If all we knew about women was what we read in \textit{Demography}, what would we know?}
\end{center}
El inter\'es principal de \cite{watkins1993if} es estudiar c\'omo se entiende a la mujer en el campo demogr\'afico, particularmente en los art\'iculos de la revista \textit{Demography}. La conclusi\'on indicar\'ia que las mujeres son, principalmente, encargadas de la producir hijos y de criarlos sin ayuda de los hombres, que est\'an aisladas de la sociedad y que el compromiso que sienten hacia los hijos y la familia se puede quebrantar a la primera oportunidad. Sin embargo, se sabe a\'un menos sobre las actividades y caracter\'isticas de los hombres, por lo que se tiene un entendimiento incompleto sobre estos fen\'omenos.\\
\\
Al analizar los art\'iculos, se observ\'o que, cuando se habla de fecundidad, matrimonio y familia; se tienen muchas preconcepciones de los roles de g\'enero que se aceptan como premisas pero sin evidencia como podr\'ian ser las expectativas de falsa informaci\'on en las encuestas, expectativas de reporducci\'on o motivos para unirse; estas concepciones pueden entorpecer el an\'alisis posterior. Con la crisis de sobrepoblaci\'on en los 60's, todos los art\'iculos y programas se dirig\'ain a la mujer como \'unico agente de soluci\'on, evidenciando lo establecido de los roles. Incluso se apel\'o a los nuevos moviemientos de empoderaci\'on femenina instando a las mujeres a salir del hogar y disminuir su reproducci\'on.\\
\\
A ver a la mujer como un veh\'iculo para la reproducci\'on, la demograf\'ia solo se encarg\'o de estudiarlas mientras fueran f\'ertiles, es decir, que una mujer antes de la menarca o despu\'es de la menopausia se consideraba in\'util y sin importancia, obviando el papel que tienen en las redes sociales de otras mujeres. En estos mismos an\'alisis se designan variables explicativas insuficientes y distintas a las de los hombres, pensando otra vez que la mujer debe saber cu\'antos hijos ha tenido mientras que se exime al var\'on de la misma responsabilidad. De este modo, se distinguen dos marcos te\'oricos con los cuales se estudia la fecundidad: la modernizaci\'on, un enfoque m\'as vago con el cual se aborda la transformaci\'on social asociada con la industrializaci\'on y las nuevas ideas; o el NHE (Nueva Econom\'ia del Hogar), que est\'a m\'as enfocado en un conjunto de ideas relacionados con la asignaci\'on eficiente del tiempo y los recursos dentro y fuera del hogar. Ambos enfoques comparten su asiganci\'on de roles con la mujer a cargo del trabajo dom\'estico y el hombre del trabajo extradom\'estico. En estos enfoques, se consideran las din\'amicas de poder involucradas en la decisi\'on de reproducci\'on, conferiendo mayor autoridad expl\'icita a los hombres y se plantea la idea de que el compromiso de la mujer a la esfera dom\'estica solo se mantiene mientras que no reciba mayor educaci\'on o incentivos para salir de ah\'i, por lo que permanece la pregunta de qui\'en cuida a los ni$\tilde{n}$os y mantiene la esfera dom\'estica.\\
\\
De nuevo se menciona el poder que tiene la ideolog\'ia y la cultura en la manera en qu\'e se hace ciencia y se analizan los fen\'omenos demogr\'aficos. Si no se reconoce esta falta de objetividad de parte de los investigadores, ser\'a imposible tener un entendimiento integral de las din\'amicas de poblaci\'on y se seguiran generando proyecciones insuficientes. 
\bibliographystyle{apalike}
\bibliography{bib_COLMEX}
\end{document}
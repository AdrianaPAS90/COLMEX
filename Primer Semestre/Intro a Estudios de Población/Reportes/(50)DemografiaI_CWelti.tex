\documentclass[11pt,spanish,letterpaper]{article}
\usepackage[spanish,activeacute]{babel}
\usepackage{caption}
\usepackage{subcaption}
\usepackage[latin1]{inputenc}
\usepackage{babel}
\usepackage{picture}
\usepackage{url}
\usepackage{mathrsfs}
\usepackage{amssymb,amsthm,amsmath,latexsym}
\usepackage[round]{natbib}
\usepackage{fancyhdr}
\theoremstyle{plain}
\newtheorem{teo}{Teorema}
\newtheorem{prop}[teo]{Proposicion}
\newtheorem{defi}[teo]{Definicion}
\newtheorem{obs}[teo]{Observacion}
\newtheorem{lem}[teo]{Lema}
\newtheorem{cor}[teo]{Corolario}
\usepackage[pdftex]{color,graphicx}
\newcommand{\sgn}{\mathop{\mathrm{sgn}}}
\setlength{\textwidth}{12.6cm}
\setlength{\textheight}{19cm}
\begin{document}
\begin{flushleft}
Adriana P\'erez-Arciniega Sober\'on
\end{flushleft}
\begin{center}
\textbf{Especificaci\'on de la Demograf\'ia y relaci\'on con las ciencias sociales}
\end{center} 
El primer cap\'itulo de \cite{welti1997demografia} trata de definir la demograf\'ia como el estudio de la poblaci\'on, enfoc\'andose de manera espec\'ifica en la estructura, la din\'amica y los componentes de la misma, la fecundidad, la mortalidad y la migraci\'on. En el contexto demogr\'afico, la estructura se refiere a la distribuci\'on de la poblaci\'on de acuerdo a aspecto espec\'ificos como la edad o el sexo. En la poblaci\'on se pueden definir dos procesos que la afectan directamente: los procesos de entrada y salida. Los procesos de entrada se refieren a los nacimientos e inmigraciones y los procesos de salida a las muertes y a las emigraciones. La demograf\'ia se dedica a explorar como estos procesos afectan la comentada estructura de poblaci\'on.\\
\\
Tambi\'en el cap\'itulo se enfatiza la relaci\'on de la demograf\'ia con las otras ciencias sociales mediante un repaso de la historia mundial de las poblaciones. De este modo, se establece una relaci\'on de intensa cooperaci\'on entre ambos bandos resultando en la inclusi\'on de la problem\'atica de la estructura y cambio poblacional en estudios sociales multidisciplinarios, aportando nuevas perspectivas de an\'alisis a problemas de otras disciplinas y ayudando a mayor precisi\'on en la planificaci\'on de pol\'iticas p\'ublicas.\\
\\
En esta revisi\'on de la historia mundial de poblaci\'on se identifican tres grandes ciclos del aumento demogr\'afico, la Revoluci\'on Agraria, la Edad Media y la Revoluci\'on Industrial. Estos cambios en las poblaciones han derivado en distintas corrientes de pensamiento. En la antig$\ddot{u}$edad se promovi\'ia la idea de una alta fecundidad para prop\'ositos econ\'omicos, militares y de sobrevivencia. El pensamiento moderno es aquel que inicia con la ideolog\'ia capitalista diciendo que el crecimiento de la poblaci\'on es favorable al incremento en los mercados de oferta y demanda. Otra corriente de pensaminto importante es aquella de la aritm\'etica demogr\'afica desarrollada por Graunt en 1662 con los c\'alculos de porcentajes de mortalidad y natalidad, seguido por Halley que construye la primera tabla de mortalidad; mientras que Cantillion elabor\'o la teor\'ia del cambio en comportamientos de fecundidad dependiendo del estrato social; Malthus afirm\'o que la miseria era resultado de un desequilirio entre la poblaci\'on y sus recursos, por lo que lo \'unico que se pod\'ia hacer era descubrir nuevas tierras f\'ertiles. A$\tilde{n}$os despu\'es, Pearl encontr\'o una curva matem\'atica que describe de manera acertada el comportamiento cuantitativo de las poblaciones humanas. El pensamiento socialista tambi\'en desarroll\'o sus teor\'ias de poblaci\'on rechazando las hip\'otesis malthusianas afirmando que los medios de produccci\'on del socialismo no puede generar un excedente de poblaci\'on.\\
\\
Las teor\'ias de poblaci\'on se han enfocado en distintos ejes a lo largo de la historia, es por eso que en el siglo pasado se delinean las siguientes ramas de la demograf\'ia. En primer lugar, la demograf\'ia formal, que sigue la tendencia de la artim\'etica demogr\'afica enfocandose en las relaciones anal\'iticas entre los componentes de la din\'amica de poblaciones y formulando la teor\'ia de las poblaciones estables. En segundo lugar, el enfoque de poblaci\'on y desarrollo, el cual se dedica a estudiar las relaciones entre las variables de la poblaci\'on y el desarrollo social y econ\'omico. Por \'ultimo, como una rama que concilia las dos anteriores est\'a el estudio espec\'ifico de los componentes de las din\'amicas de poblaci\'on, donde se asume la necesidad de establecer relaciones anal\'iticas y cuantitativas mientras se buscan las determinaciones sociales, econ\'omicos y culturales de cada componente de los procesos demogr\'aficos.\\
\bibliographystyle{apalike}
\bibliography{bib_COLMEX}
\end{document}
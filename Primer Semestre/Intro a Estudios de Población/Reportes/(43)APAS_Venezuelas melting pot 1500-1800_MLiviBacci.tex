\documentclass[11pt,spanish,letterpaper]{article}
\usepackage[spanish,activeacute]{babel}
\usepackage{caption}
\usepackage{subcaption}
%\usepackage[latin1]{inputenc}
\usepackage{babel}
\usepackage{picture}
\usepackage{url}
\usepackage{mathrsfs}
\usepackage{amssymb,amsthm,amsmath,latexsym}
\usepackage[round]{natbib}
\usepackage{fancyhdr}
\theoremstyle{plain}
\newtheorem{teo}{Teorema}
\newtheorem{prop}[teo]{Proposicion}
\newtheorem{defi}[teo]{Definicion}
\newtheorem{obs}[teo]{Observacion}
\newtheorem{lem}[teo]{Lema}
\newtheorem{cor}[teo]{Corolario}
\usepackage[pdftex]{color,graphicx}
\newcommand{\sgn}{\mathop{\mathrm{sgn}}}
\setlength{\textwidth}{12.6cm}
\setlength{\textheight}{19cm}
\begin{document}
\begin{flushleft}
Adriana P\'erez-Arciniega Sober\'on
\end{flushleft}
\begin{center}
\textbf{Venezuela’s melting pot: 1500-1800}
\end{center}
En el art\'iculo de \cite{bacci2017venezuela} comienza describiendo como la poblaci\'on venezolana previa a ser colonizada por Espa$\tilde{n}$a es desconocida en mayor parte. No es hasta los registros religiosos de las \'ordenes que fueron encargadas de evangelizar a la poblaci\'on nativa, que se pudo hacer un primer estimado sobre la poblaci\'on ind\'igena. Con ayuda de m\'etodos modernos se ha podido estimar que de una poblaci\'on inicial de 200 mil a 500 mil, esta se redujo a 120 mil para 1800; la primera gran reducci\'on de la poblaci\'on fue producto de una gran epidemia de viruela alrededor de la d\'ecada de 1580.\\
\\
En la \'epoca de la colonia en Venezuela se observaron los mismos comportamientos que en el resto de Am\'erica del sur, la nupcialidad ocurr\'ia a edades tempranas y era casi universal, mientras que los altos \'indices de mortalidad y fecundidad muestra un gran potencial de crecimiento. Otra de las causas de reducci\'on de la poblaci\'on nativa fue el alto grado de mestizaje que ocurri\'o en la poblaci\'on, t\'ipico de los pa\'ises coloniales.\\
\\
Uno de ls grandes historiadores de las colonias espa$\tilde{n}$olas, Alexander Von Humboldt, desembarc\'o en costas venezolanas en 1799 y estim\'o que para esos momentos la poblaci\'on se pod\'ia estimar en 800 mil habitantes, de los cuales 120 mil eran ind\'igenas, 200 mil de origen espa$\tilde{n}$ol o criollos, 60 o 70 mil eran esclavos de origen africano y alrededor de 400 mil eran habitantes mestizos. Aunque no hay manera de conocer la poblaci\'on cuando Col\'on lleg\'o al continente, mediante estimaciones modernas, los c\'alculos oscilan entre 200 mil y 500 mil habitantes.\\
\\
Como sucedi\'o con el resto de las colonias, la llegada de los espa$\tilde{n}$oles, merm\'o significativamente la poblaci\'on ind\'igena, que en el caso de Venezuela significaban escasos asentamientos de poblaci\'on semi n\'omada, dispersa en tribus; a diferencia de otras partes de Mesoam\'erica. Los colonizadores espa$\tilde{n}$oles introdujeron la ganader\'ia y la agricultura para establecer sistemas de comercio y gobierno.\\
\\
Los factores que afectaron la composici\'on demogr\'afica de la Venezuela colonial son tres: patolog\'ias,  violencia y esclavitud. Muchos estudios consideran que las patolog\'ias importadas de Eurasia fueron un gran factor en la disminuci\'on de la poblaci\'on ind\'igena, una de estas patolog\'ias fue la viruela. Las consecuencias demogr\'aficas de una epidemia, adem\'as de la reducci\'on de la poblaci\'on, es la inoculaci\'on de las personas que sobrevivieron a la enfermedad, el desarrollo de un proceso de aprendizaje social sobre el cuidado de los enfermos y, que despu\'es de una epidemia, el n\'umero de matrimonios y nacimietos aumentaba. El factor de la violencia se refiere en primer lugar al exterminio de la poblaci\'on nativa, al igual que la esclavitud.\\
\\
Los efectos negativos de la conquista espa$\tilde{n}$ola que provocaron la esclavitud, las cargas inhumanas de trabajo y el desplazamiento de poblaciones ind\'igenas de sus asentamientos a regiones m\'as inh\'ospitas. Aunado a la introducci\'on de diversas patolog\'ias, la poblaci\'on ind\'igena se reduce en gran medida, de acuerdo con Humbolt a 120 mil nativos. Adem\'as, la poblaci\'on ind\'igenas sobreviviente sigui\'o reduciendose de acuerdo al mestizaje, a\'un despu\'es de la independencia de Venezuela como colonia.
\bibliographystyle{apalike}
\bibliography{bib_COLMEX}
\end{document}
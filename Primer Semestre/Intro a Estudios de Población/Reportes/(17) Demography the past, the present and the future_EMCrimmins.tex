\documentclass[11pt,spanish,letterpaper]{article}
\usepackage[spanish,activeacute]{babel}
\usepackage{caption}
\usepackage{subcaption}
\usepackage[latin1]{inputenc}
\usepackage{babel}
\usepackage{picture}
\usepackage{url}
\usepackage{mathrsfs}
\usepackage{amssymb,amsthm,amsmath,latexsym}
\usepackage[round]{natbib}
\usepackage{fancyhdr}
\theoremstyle{plain}
\newtheorem{teo}{Teorema}
\newtheorem{prop}[teo]{Proposicion}
\newtheorem{defi}[teo]{Definicion}
\newtheorem{obs}[teo]{Observacion}
\newtheorem{lem}[teo]{Lema}
\newtheorem{cor}[teo]{Corolario}
\usepackage[pdftex]{color,graphicx}
\newcommand{\sgn}{\mathop{\mathrm{sgn}}}
%\author{Adriana Perez-Arciniega Soberon} 
\begin{document}
\title{Demography: The Past 30 years, the present and the future}
\date{}
\maketitle
A diferencia de otros autores, \cite{crimmins1993demography} define a la demograf\'ia como un campo multifac\'etico dentro de una disciplina, as\'i que determinar sus intereses de investigaci\'on dependen m\'as bien de en qu\'e faceta se enfatiza. Este art\'iculo busca definir el campo futuro de investigaci\'on , por lo que busca ayudarse de las cuatro facetas del campo: c\'omo y d\'onde se hace demograf\'ia, el tipo de datos que se utlizan, los m\'etodos que se usan y el enfoque te\'orico y las preguntas que se buscan contestar.\\
\\
En lo que se refiere a c\'omo se hace demograf\'ia, se pone mucho \'enfasis en los cambios de tecnolog\'ia de procesamiento de informaci\'on, facilitando el manejo de grandes bases de datos a un costo menos, mejorando as\'i el an\'alisis y la creaci\'on de modelos te\'oricos m\'as precisos y basados en la realidad. Debido a la facilidad que da el cambio tecnol\'ogico, esto propiciar\'a que los estudiosos de la demograf\'ia se junten para estudiar el mismo tema desde diferentes \'angulos, logrando una alta especializaci\'on. De la mano con el incremento en tecnolog\'ia, los datos se han refinido en los \'ultimos treinta a$\tilde{n}$os, al principio el an\'alisis se hac\'ia con datos agregados para estudiar comportamientos espec\'ificos; pero con el tiempo, tambi\'en la manera en que se obtuvieron los datos se volvi\'o m\'as precisa con el desarrollo de encuestas transversales, que satisfacen mejor las demandas de informaci\'on del campo actual.\\
\\
En cuanto a un an\'alisis de metodolog\'ia, la demograf\'ia formal ha tenido cierta continuidad debido a las preguntas que le conciernen sobre las estructuras de poblaci\'on y c\'omo estas se relacionan con los procesos demogr\'aficos de fertilidad, mortalidad y migraci\'on, pero recientemente se han incluido m\'etodos desarrollados por estad\'isticos, epidemi\'ologos, entre otros; logrando que se evolucione una visi\'on esructuralista a una m\'as enfocada en los procesos. Esto ha logrado tambi\'en un cambio en el enfoque te\'orico de la demograf\'ia, hace treinta a$\tilde{n}$os se buscaba describir la mortalidad, natalidad, migraci\'on y fuerza de trabajo, mientras que, recientemente, estas son solo unas variables m\'as de la pregunta general. Ahora, se busca inclur las relaciones socio-econ\'omicas y demogr\'aficas para explicar los procesos demogr\'aficos.\\
\\
En conclusi\'on, la demograf\'ia ha cambiado mucho en los \'ultimos treinta a$\tilde{n}$os, debido a los cambios tecnol\'ogicos que han permitido desarrollar nuevas metodolog\'ias, ampliar preguntas de investigaci\'on y reconfigurar el panorama del campo. Sin embargo, persiste el mismo n\'ucleo de intereses y temas.
\bibliographystyle{apalike}
\bibliography{bib_COLMEX}
\end{document}
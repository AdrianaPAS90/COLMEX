\documentclass[11pt,spanish,letterpaper]{article}
\usepackage[spanish,activeacute]{babel}
\usepackage{caption}
\usepackage{subcaption}
%\usepackage[latin1]{inputenc}
\usepackage{babel}
\usepackage{picture}
\usepackage{url}
\usepackage{mathrsfs}
\usepackage{amssymb,amsthm,amsmath,latexsym}
\usepackage[round]{natbib}
\usepackage{fancyhdr}
\theoremstyle{plain}
\newtheorem{teo}{Teorema}
\newtheorem{prop}[teo]{Proposicion}
\newtheorem{defi}[teo]{Definicion}
\newtheorem{obs}[teo]{Observacion}
\newtheorem{lem}[teo]{Lema}
\newtheorem{cor}[teo]{Corolario}
\usepackage[pdftex]{color,graphicx}
\newcommand{\sgn}{\mathop{\mathrm{sgn}}}
\setlength{\textwidth}{12.6cm}
\setlength{\textheight}{19cm}
\begin{document}
\begin{flushleft}
Adriana P\'erez-Arciniega Sober\'on
\end{flushleft}
\begin{center}
\textbf{Historical demography: achievements and prospects}
\end{center}
\cite{saito1996historical} resalta la importancia de la demograf\'ia hist\'orica a trav\'es de la \'optica francesa. Una de las principales herramientas de la inciativa francesa fue la reconstrucci\'on familiar. Este m\'etodo fue muy popular en la escuela francesa e inglesa a mediados de los a$\tilde{n}$os 60's, en gran medida por lo bien que se adapt\'o a los estudios de fecundidad, aunque no era tan efectivo para el estudio de la mortalidad o para el estudio de migraci\'on.\\
\\
Al principio, el investigador Hajnal introdujo el tema de sistema de formaci\'on familiar basado en supuestos sobre nupcialidad y estructura de los hogares para determinar otros fen\'omenos como la fecundidad. Esta teor\'ia se complement\'o con lo propuesto por Laslett, en lo denominado, servidores de ciclo de vida para poder entender la din\'amica demogr\'afica de la Inglaterra preindustrial. Mientras que las aportaciones de McKeown en los estudios de mortalidad fueron cruciales para lograr un nuevo enfoque, al determinar que las condiciones de salud no depend\'ian de manera directa de innovaciones en la medicina.\\
\\
De este modo, se puede determinar que los demogr\'afos hist\'oricos deben de desarrollar t\'ecnicas de an\'alisis m\'as finas, sobre todo si se busca estudiar distintas poblaciones con diferentes interpretaciones de fen\'omenos demogr\'aficos como la reproducci\'on o la nupcialidad. Aunado al desarrollo de estas t\'ecnicas se podr\'ia hablar de ampliar la base de fuentes de datos que se utilizan, debido que fuentes como registros parroquiales est\'an subutilizados y son de gran utilidad en entender el contexto cultural de las sociedades a estudiar.\\
\\
Con la utilizaci\'on de nuevas fuentes de datos sobre din\'amicas demogr\'aficas del pasado sumado a robustas t\'ecnicas estad\'isticas, se podr\'ia fortalecer el m\'etodo de reconstrucci\'on familiar. Tambi\'en es importante conocer la historia econ\'omica y social de la poblaci\'on a estudiar, como ha sido la historia del mercado para entender la transici\'on demogr\'afica en Europa. Es por esto que, aunque se encuentre al pensamiento puramente demogr\'afico fascinante, para poder entender y estudiar de la manera m\'as adecuada la historia de las poblaciones, este estudio debe realizarse desde una perspectiva interdisciplinaria.
\bibliographystyle{apalike}
\bibliography{bib_COLMEX}
\end{document}
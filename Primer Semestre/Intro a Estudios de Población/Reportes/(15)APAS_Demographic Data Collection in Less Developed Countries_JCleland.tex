\documentclass[11pt,spanish,letterpaper]{article}
\usepackage[spanish,activeacute]{babel}
\usepackage{caption}
\usepackage{subcaption}
%\usepackage[latin1]{inputenc}
\usepackage{babel}
\usepackage{picture}
\usepackage{url}
\usepackage{mathrsfs}
\usepackage{amssymb,amsthm,amsmath,latexsym}
\usepackage[round]{natbib}
\usepackage{fancyhdr}
\theoremstyle{plain}
\newtheorem{teo}{Teorema}
\newtheorem{prop}[teo]{Proposicion}
\newtheorem{defi}[teo]{Definicion}
\newtheorem{obs}[teo]{Observacion}
\newtheorem{lem}[teo]{Lema}
\newtheorem{cor}[teo]{Corolario}
\usepackage[pdftex]{color,graphicx}
\newcommand{\sgn}{\mathop{\mathrm{sgn}}}
\setlength{\textwidth}{12.6cm}
\setlength{\textheight}{19cm}
\begin{document}
\begin{flushleft}
Adriana P\'erez-Arciniega Sober\'on
\end{flushleft}
\begin{center}
\textbf{Demographic data collection in less developed countries 1946--1996}
\end{center}
\cite{cleland1996demographic} relata como antes de la d\'ecada de 1950 no exist\'ian estudios demogr\'aficos en los pa\'ises en v\'ias de desarrollo, por lo que en primer lugar se ocuparon de los m\'etodos de recolecci\'on de datos. En estos pa\'ises la recolecci\'on de datos a\'un se daba por un proceso de preguntas y respuestas sobre fen\'omenos demogr\'aficos como la mortalidad o los nacimientos. Estos m\'etodos son especialmente efectivos para el registro de el n\'umero de hijos nacidos vivos por mujer a lo largo de su vida y en los \'ultimos 12 meses; sin embargo, conceptos como nupcialidad han sido suhetos a cambios sociales en los per\'iodos pre-guerra en pa\'ises africanos. Aunque los principales m\'etodos de recolecci\'on de datos ya estaban establecidos a trav\'es de censos y registros administrativos, se ha buscado el refinamiento de m\'etodos que permitan contar  con informaci\'on m\'as precisa.\\
\\
Los censos en s\'i mismos, eran un elemento problem\'atico, pues siendo la principal fuente de informaci\'on estos se ve\'ian muy espaciados en el tiempo y con relativamente mala calidad de informaci\'on . Mediante incrementos en alfabetizaci\'on y avances en educaci\'on, se lograban mejores mediciones aunque contrarrestadas por la mayor movilidad geogr\'afica de los habitantes. Estas fallas de inicio en los procesos de recolecci\'on han dado lugar a muchas controversias en el campo de los estudios de poblaci\'on: por un lado, se argumentaba la necesidad de mejorar la calidad de los datos recolectados desde el inicio mediante el dise$\tilde{n}$o de procesos de medici\'on en el campo; mientras que otro lado argumenta que se deber\'ia mantener los datos como estaban e invertir en m\'etodos de ajuste y captaci\'on indirecta. Independientemente, se desarrollaron m\'etodos para derivar estimaciones a partir de los datos defectuosos con los que se contaba mediante dise$\tilde{n}$os prospectivos.\\
\\
Uno de los primeros mecanismos que se implementaron para aumentar la calidad de los datos fueron las visitas m\'ultiples, en las cuales un mismo entrevistador visitaba un hogar varias veces en un per\'iodo de tiempo pod\'ia determinar \'el mismo los cambios en la estructura del hogar. Esto resulto ser ineficiente por los recursos que deb\'ian destinarse a estas visitas, por lo que se implement\'o el registro dual, donde se registraba la poblaci\'on mediante encuestas y regstradores locales y despu\'es se cotejaban ambas listas; sin embargo, este m\'etodo era vulnerable al criterio del registrador local. La mejor forma que se ha encontrado de recolectar esta informaci\'on es mediante encuestas, principalmente la Encuesta Mundial de Fecundidad (WFS) y la Encuesta Demogr\'afica de Salud (DHS), con una calidad sorprendentemente buena. As\'i se concluye que la calidad de los datos demogr\'aficos depende de las habilidades, capacitaci\'on y supervisi\'on del personal que recolecta los datos que en el dise$\tilde{n}$o de los instrumentos de captaci\'on.\\
\\
En conclusi\'on, se deben tomar en cuenta las necesidades para las cu\'ales la informaci\'on se va a recolectar. Estas necesidades pueden ser de dos categor\'ias: planificaci\'on de pol\'iticas de desarrollo para la cual se necesitan datos socioecon\'omicos, su estructura y su distribuci\'on geogr\'afica; la segunda es la evaluaci\'on de programas para influir en las tasas vitales. Estos m\'etodos de recolecci\'on son suficientes actualmente aunque quiz\'a en un futuro esto se necesite complementar con m\'etodos biom\'edicos.
\bibliographystyle{apalike}
\bibliography{bib_COLMEX}
\end{document}
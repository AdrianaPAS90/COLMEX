\documentclass[11pt,spanish,letterpaper]{article}
\usepackage[spanish,activeacute]{babel}
\usepackage{caption}
\usepackage{subcaption}
%\usepackage[latin1]{inputenc}
\usepackage{babel}
\usepackage{picture}
\usepackage{url}
\usepackage{mathrsfs}
\usepackage{amssymb,amsthm,amsmath,latexsym}
\usepackage[round]{natbib}
\usepackage{fancyhdr}
\theoremstyle{plain}
\newtheorem{teo}{Teorema}
\newtheorem{prop}[teo]{Proposicion}
\newtheorem{defi}[teo]{Definicion}
\newtheorem{obs}[teo]{Observacion}
\newtheorem{lem}[teo]{Lema}
\newtheorem{cor}[teo]{Corolario}
\usepackage[pdftex]{color,graphicx}
\newcommand{\sgn}{\mathop{\mathrm{sgn}}}
\setlength{\textwidth}{12.6cm}
\setlength{\textheight}{19cm}
\begin{document}
\begin{flushleft}
Adriana P\'erez-Arciniega Sober\'on
\end{flushleft}
\begin{center}
\textbf{Historical population estimates: Unraveling the consensus}
\end{center}
En el art\'iculo de \cite{caldwell2002historical} se trata de entender la historia del consenso sucedido a mediados del siglo XX del crecimiento poblacional hasta 1650. Este consenso se bas\'o en dos publicaciones , una de Walter Willcox y la otra de Alexander Carr-Saunders, que fueron avaladas por la Organizaci\'on de las Naciones Unidas y de este modo, consolid\'o el consenso. Las principales cr\'iticas al consenso eran que ambos art\'iculos se confirmaban entre ellos y dando lugar a cierta circularidad en los argumentos, que no exist\'ian fuentes confiables de datos antes de 1750 o que los intervalos de las estimaciones eran muy grandes.\\
\\
El art\'iculo tambi\'en trata de hacer \'enfasis en las diferencias en las motivaciones de los dem\'ografos hist\'oricos y de los que proyectan poblaciones futuras. Los que hacen proyecciones futuras se enfocan en n\'umeros totales de poblaciones futuras y sustentabilidad, mientras que los dem\'ografos hist\'oricos se enfocan en el cambio poblacional m\'as que en el equilibrio entre poblaci\'on y recursos, dado que se toma como un bloque de construcci\'on a la sociedad moderna.\\
\\
De los autores del consenso, solo Willcox menciona fuentes de datos. Las fuentes de datos que normalmente usar\'ian los dem\'ografos hist\'oricos estar\'ian incompletos al ser relatos orales o en otro idioma, as\'i que Willcox se bas\'o en estudios alemanes del siglo XIX de un te\'ologo llamado Johann Peter Süssmiclh que trataba de buscar patrones en las poblaciones que reflejaran a Dios, sumados a estudios de Karl Friedrich Dieterici en el cual se investigaron los censos disponibles, encuestas y reportes de viaje. Sin embargo, en muchas partes del mundo no se contaron con censos hasta finales del siglo XIX, por lo que las estimaciones en dichas partes del mundo se basaban en cr\'onicas de viaje o recuentos b\'iblicos.\\
\\
De este modo, Willcox pudo estimar el crecimiento poblacional y concluy\'o que desde 1650 la poblaci\'on hab\'ia aumentado en 40\%. Hubo cierto debate en el tama$\tilde{n}$o y crecimiento poblacional en Asia para ese per\'iodo de tiempo, por lo que se termin\'o concluyendo que la poblaci\'on de China, Jap\'on e India en 1930, se le sumaba un tercio de su poblaci\'on se podr\'ia obtener el total de la poblaci\'on asi\'atica en 1650. Igualmente, se discuti\'o los consensos europeos debido a que se hab\'ia planteado que la poblaci\'on europea hab\'ia sido de 100 millones en 1650, 140 millones en 1750 y 187 millones en 1800; sin embargo, tomando en cuenta las tasas de crecimiento planteadas se tendr\'ia que la poblaci\'on europea en 1650 deb\'ia de ser de 116 millones y aunque se pudiera explicar, de cualquier modo se pone en duda toda la teor\'ia del crecimiento europeo.\\
\\
Pero el consenso que caus\'o m\'as debate fue el de \'Africa. Debido que para el consenso de 1650, se consideraba que la poblaci\'on africana constitu\'ia un quinto de la poblaci\'on mundial. Aunque Willcox y Carr-Saunders fueron refinando sus estimaciones para Am\'erica y Europa conforme fueron surgiendo mejores fuentes de informaci\'on, esto no paso con \'Africa. El principal problema era la precariedad de las fuentes de informaci\'on, pues ni siquiera los censos del siglo XX son totalmente confiables, es por esto que Willcox y Carr-Saunders siguieron tomando como base las estimaciones de Riccoli realizadas en el siglo XIX argumentando que la poblaci\'on africana segu\'ia en una fase cultural m\'as primitiva, sin un sistema agrario bien establecido y que estaba sujeta a procesos de home\'ostasis.\\
\\
A principios del siglo XX, se segu\'ia considerando que la poblaci\'on africana segu\'ia siendo regulada mediante la home\'ostasis, sin tomar en cuenta los efectos demogr\'aficos que tuvieron la colonizaci\'on europea. Con la introducci\'on de nuevos cultivos y nuevos m\'etodos agr\'icolas, cambiaron los fen\'omenos demoggr\'aficos africanos. Sin embargo, Carr-Saunders segu\'ia manteniendo la cifra de 100 millones de habitantes en \'Africa, debido a la creencia de que cualquier efecto de incremento poblacional derivado de la colonizaci\'on europea, era constrarrestado con el efecto de la esclavitud. Un estudio de 1969 neg\'o la importancia de este efecto, determinando que el volumen de esclavitud era en realidad la mitad de lo que cre\'ia Carr-Saunders, aunado a que la presi\'on poblacional era m\'as lenta por la introducci\'on de nuevos cultivos adem\'as que solo un tercio de la poblaci\'on esclavizada era femenina, por lo que quedaban muchas mujeres en el continente para mantener la poblaci\'on a largo plazo. As\'i, ahora se puede concluir que el decrecimiento poblacional en \'Africa no hab\'ia sucedido y que la poblaci\'on africana para 1650 era mucho menor de lo proyectado por Willcox y Carr-Saunders.\\
\\
En la segunda mitad del siglo XX, ha surgido una nueva oleada de dem\'ografos interesados en realizar estimaciones poblacionales en distintos momentos del pasado. Aunque en su mayor\'ia han concordado con lo planteado por Willcox y Carr-Saunders, el inter\'es ha sido en realizar estimaciones en per\'iodos m\'as antiguo, de 0-1500 D.C. con la ayuda de ge\'ografos, determinando las fronteras agr\'icolas y las densidades poblacionales en \'areas rurales. Lo que s\'i se derrumb\'on en esta nueva oleada fue el consenso africano, estimando la poblaci\'on de 1500 en 50 millones y reduciendo la estimaci\'on moderna en 16.5 millones.\\
\\
Las estimaciones para per\'iodos previos al siglo XVII nunca podr\'an ser totalmente precisas por la falta de fuentes de informaci\'on, soin embargo, sigue siendo un ejercicio valioso dado que relaciona el tama$\tilde{n}$o de las poblaciones a otros fen\'omenos y fuerza al investigador a poner su informaci\'on en contexto, como los distintos tipos de cosechas, el uso de suelo o el nivel de urbanizaci\'on de la poblaci\'on que se estudia.\\
\\
En conclusi\'on, el consenso no es mas que el directo resultado del trabajo de Willcox, de Carr-Saunders y de las fuentes alemanas utilizadas. El problema con el consenso es el peso que conlleva para los nuevos investigadores y que puede derivar en una circularidad en el pensamiento y termine siendo en perjuicio de la misma disciplina.
\bibliographystyle{apalike}
\bibliography{bib_COLMEX}
\end{document}
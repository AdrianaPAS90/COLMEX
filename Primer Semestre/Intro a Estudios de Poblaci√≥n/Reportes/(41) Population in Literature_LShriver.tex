\documentclass[11pt,spanish,letterpaper]{article}
\usepackage[spanish,activeacute]{babel}
\usepackage{caption}
\usepackage{subcaption}
\usepackage[latin1]{inputenc}
\usepackage{babel}
\usepackage{picture}
\usepackage{url}
\usepackage{mathrsfs}
\usepackage{amssymb,amsthm,amsmath,latexsym}
\usepackage[round]{natbib}
\usepackage{fancyhdr}
\theoremstyle{plain}
\newtheorem{teo}{Teorema}
\newtheorem{prop}[teo]{Proposicion}
\newtheorem{defi}[teo]{Definicion}
\newtheorem{obs}[teo]{Observacion}
\newtheorem{lem}[teo]{Lema}
\newtheorem{cor}[teo]{Corolario}
\usepackage[pdftex]{color,graphicx}
\newcommand{\sgn}{\mathop{\mathrm{sgn}}}
%\author{Adriana Perez-Arciniega Soberon} 
\begin{document}
\title{Population in Literature}
\date{}
\maketitle
El texto de \cite{shriver2003population}, al hablar de las inquietudes demogr\'aficas, plantea dos principales: las relacionadas a la fecundidad y a la mortalidad en el contexto de grandes obras literarias. De este modo, se resaltan a el arte literario como un medio de comunicaci\'on eficiente y con poder de convencimiento, para llevar los problemas demogr\'aficos al imaginario com\'un.\\
\\ 
Cuando las tasas de fecundidad comenzaron a caer varias obras literarias como \textit{F\'econdit\'e} por Emile Zola o \textit{Headbirths or The Germans are dying out} de Gunter Grass, se dieron a la tarea de describir un futuro apocal\'iptico, en el cual no existir\'ian ni$\tilde{n}$os destinado a la extinci\'on. Adem\'as, en muchas de estas obras se exalta la superioridad moral de aquellos quienes tiene hijos.\\
\\
En cuanto a novelas que exploran la mortalidad, a pesar de algunas excepciones con tintes de ciencia ficci\'on, como \textit{The Machine Stops} o cat\'astrofes naturales como \textit{The Long Winter}; la mayor\'ia de estas obras tratan sobre guerras o enfermedades. Las obras de guerra se concentraron en momentos donde hab\'ia una preocupaci\'on en el mundo real sobre esta \'indole como la Segunda Guerra Mundial o la Guerra Fr\'ia como con el libro \textit{On the Beach} o \textit{Cat's Cradle}. De la misma manera, novelas sobre la extinci\'on de la civilizaci\'on provocada por enfermedades tuvo un crecimiento con el desarrollo y aumento de diagn\'osticos de VIH/SIDA; con esta tem\'atica podemos encontrar novelas como \textit{After the Plague} o \textit{Earth Abides}.\\
\\
Por otro lado, existe la literatura que advierte de los peligros de la sobrepoblaci\'on, ya sea inducida por el envejecimiento y aumento en la esperanza de vida, o por un aumento en las tasas de fecundidad. En t\'erminos de 'inmortalidad' existe la historia \textit{Tomorrow, Tomorrow, Tomorrow} y el aumento en las tasas de fecundidad en el libro \textit{Make Room!, Make Room!}.\\
\\
Todas estas obras no solo hablan de los peligros de una sobrepoblaci\'on o de la extinci\'on de la raza humana, sino de las consecuencias morales y sociales que esto conllevar\'ia. Esto indica que el inter\'es por lo temas demogr\'aficos existen en el arte y en la comunidad, sin embargo, las obras de arte toman dos funciones distintas. La primera como herramienta propagand\'istica para impulsar la estrategia de poblaci\'on conveniente o bien, para explorar las emociones humanas que surgen de los problemas de poblaci\'on.
\bibliographystyle{apalike}
\bibliography{bib_COLMEX}
\end{document}
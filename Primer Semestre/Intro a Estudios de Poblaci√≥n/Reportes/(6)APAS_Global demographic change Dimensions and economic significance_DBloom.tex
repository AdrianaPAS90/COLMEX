\documentclass[11pt,spanish,letterpaper]{article}
\usepackage[spanish,activeacute]{babel}
\usepackage{caption}
\usepackage{subcaption}
%\usepackage[latin1]{inputenc}
\usepackage{babel}
\usepackage{picture}
\usepackage{url}
\usepackage{mathrsfs}
\usepackage{amssymb,amsthm,amsmath,latexsym}
\usepackage[round]{natbib}
\usepackage{fancyhdr}
\theoremstyle{plain}
\newtheorem{teo}{Teorema}
\newtheorem{prop}[teo]{Proposicion}
\newtheorem{defi}[teo]{Definicion}
\newtheorem{obs}[teo]{Observacion}
\newtheorem{lem}[teo]{Lema}
\newtheorem{cor}[teo]{Corolario}
\usepackage[pdftex]{color,graphicx}
\newcommand{\sgn}{\mathop{\mathrm{sgn}}}
\setlength{\textwidth}{12.6cm}
\setlength{\textheight}{19cm}
\begin{document}
\begin{flushleft}
Adriana P\'erez-Arciniega Sober\'on
\end{flushleft}
\begin{center}
\textbf{Global demographic change: Dimensions and economic significance}
\end{center} 
\cite{bloom2004global} describe los cambios demogr\'afcos y los efectos que han tenido en la econom\'ia en la segunda mital del siglo XX. Uno de los m\'as notables cambio en el desarrollo poblacional del siglo pasado fueron las mejoras de salud y, como consecuencia, el aumento en la esperanza de vida. Las mejoras en los esquemas de salud p\'ublica como las vacunas, repercutieron de manera particularmente positiva en las tasas de mortalidad infantil y las mejoras en las tecnolog\'ias m\'edicas, cambios en el estilo de vida y mejoras en las dietas, lograron el aumento en la esperanza de vida.\\
\\
Con la disminuci\'on en la tasa de mortalidad infantil y con la creciente participaci\'on femenina en el mercado laboral, la tasa de fecundidad emmpez\'o a decrecer r\'apidamente. De este modo, la estructura de edad de las poblaciones empez\'o a modificarse; as\'i, tenemos una mayor proporci\'on de personas j\'ovenes debido a la supervivencia en relaci\'on a generaciones anteriores. La poblaci\'on en edades j\'ovenes de mediados del siglo pasado actualmente empiezan a entrar a su sexta d\'ecada de vida, aunado al decrecimiento en las tasas de fecundidad, esta poblaci\'on representa poco m\'as de la mitad del n\'umero de personas entre 15 y 24 a$\tilde{n}$os y se prevee que esta diferencia seguir\'a aumentando. El enevejecimiento de la poblaci\'on est\'a sucediendo tanto en los pa\'ises desarrollados como en los en v\'ias de desarrollo.\\
\\
En t\'erminos econ\'omicos se puede decir que cu\'anto m\'as espere vivir una poblaci\'on, mayor ser\'a su necesidad de ahorro. As\'i se puede establecer un v\'inculo entre el tama$\tilde{n}$o y edades de la poblaci\'on y el bienestar. Este crecimiento puede observarse bajo una \'optica malthusiana o una \'optica positiva.\\
\\
Al hablar de crecimiego poblacional, una de las primeras cosas que vienen a la mente es la teor\'ia de la transici\'on demogr\'afica, la cual describe el paso de altas tasas de fecundidad y mortalidad a bajas. Sin embargo, estos cambios no suceden de manera simult\'anea, primero disminuye la mortalidad y despu\'es la fecundidad. Esto suponiendo que las poblaciones son cerradas a la migraci\'on, pero tomando en cuenta este fen\'omeno, se altera no solo la estructura de la poblaci\'on, tambi\'en su estructura et\'area.
\bibliographystyle{apalike}
\bibliography{bib_COLMEX}
\end{document}
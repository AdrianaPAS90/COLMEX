\documentclass[11pt,spanish,letterpaper]{article}
\usepackage[spanish,activeacute]{babel}
\usepackage{caption}
\usepackage{subcaption}
%\usepackage[latin1]{inputenc}
\usepackage{babel}
\usepackage{picture}
\usepackage{url}
\usepackage{mathrsfs}
\usepackage{amssymb,amsthm,amsmath,latexsym}
\usepackage[round]{natbib}
\usepackage{fancyhdr}
\theoremstyle{plain}
\newtheorem{teo}{Teorema}
\newtheorem{prop}[teo]{Proposicion}
\newtheorem{defi}[teo]{Definicion}
\newtheorem{obs}[teo]{Observacion}
\newtheorem{lem}[teo]{Lema}
\newtheorem{cor}[teo]{Corolario}
\usepackage[pdftex]{color,graphicx}
\newcommand{\sgn}{\mathop{\mathrm{sgn}}}
\setlength{\textwidth}{12.6cm}
\setlength{\textheight}{19cm}
\begin{document}
\begin{flushleft}
Adriana P\'erez-Arciniega Sober\'on
\end{flushleft}
\begin{center}
\textbf{The Population Debate in American Popular Magazines, 1946-90}
\end{center} 
El art\'iculo \cite{wilmoth1992population} se enfoca en una extensiva revisi\'on de literatura popular que habla sobre poblaci\'on en el per\'iodo de 1946-1990. La literatura popular se refiere a revistas o cualquier otra publicaci\'on peri\'odica. Se escogi\'o analizar este tipo de literatura por dos razones: la primera, es que la literatura popular escribe sobre crecimiento poblacional con un juicio de valor a$\tilde{n}$adido pero sin evidencia que sostenga el juicio; la segunda raz\'on, es que entender la evoluci\'on de las actitudes ante el crecimiento poblacional es esencial para entender el efecto de la opini\'on popular en el modelado de pol\'iticas p\'ublicas.\\
\\
Se analiz\'o una muestra de m\'as de 1,500 art\'iculos tomados de \textit{The Reader's Guide to Periodical Lecture}, el cual es un \'indice de art\'iculos contenidos en revistas populares angloparlantes de los \'ultimos cien a$\tilde{n}$os. El an\'alisis comenz\'o en el per\'iodo post-guerra, debido a que fue cuando las condiciones de salud empezaron a mejorar disminuyendo la mortalidad; igualmente las condiciones econ\'omicas mejoraron, propiciando un incremento en las tasas mundiales de natalidad. Estas condiciones propiciaron un inter\'es en el crecimiento poblacional, el cual tuvo cinco enfoques preponderantes. Tres de ellos en contra del crecimiento poblacional, uno indiferente y uno pro crecimiento poblacional.\\
\\
El enfoque malthusiano o de l\'imites de crecimiento propone, en su versi\'on extrema, que el crecimiento poblacional amenaza con la extinci\'on de la especie debido al agotamiento de recursos naturales y, en su versi\'on moderada, que la econom\'ia no cuenta con la capacidad de absorber en el mercado laboral a la poblaci\'on adicional. El enfoque de la presi\'on poblacional o sistemas mundiales se refiere a la amenaza de conflictos b\'elicos globales derivados de la presi\'on de poblaciones por expandir sus fronteras nacionales en b\'usqueda de espacios vitales. El enfoque de calidad de vida o superpoblaci\'on es uno de lo que tom\'o m\'as fuerza en el \'ultimo cuarto del siglo XX, se preocupa porque el r\'apido crecimiento poblacional amenaza la calidad de vida humana con tr\'afico, contaminaci\'on, falta de vivienda, etc. Tomando este enfoque de manera local a EE UU, como fue desarrollada en ese momento, la inmigraci\'on se considera una causa muy importante del crecimiento poblacional, por lo que se dio una justificaci\'on ideol\'ogica a movimientos nacionalistas y anti-inmigraci\'on.\\
\\
Por el otro lado, el enfoque neutral es el llamado el enfoque de suicidio de la raza o disminuci\'on de la poblaci\'on habla de la diferencia en fertilidad entre sub-grupos de poblaciones m\'as grandes, influenciados por flujos migratorios o la amenaza de dominaci\'on por alguna poblaci\'on con crecimiento m\'as r\'apido; este enfoque es neutro, pues depende de la poblaci\'on que lo est\'e viviendo. Por \'ultimo, el enfoque positivo es el crecimiento es bueno y estipula que el crecimiento poblacional es positivo para el desarrollo econ\'omico, al crear nuevos mercados y consumidores, y  para el desarrollo humano, pues se promueve descubrimientos tecnol\'ogicos con la excusa de que los humanos deben sobreponerse a los efectos a corto plazo de la sobrepoblaci\'on.\\
\\
Es importante mencionar, que todos los art\'iculos analizados fueron publicados en revistas estadounidenses, por lo que la mayor\'ia se refer\'ia a su contexto demogr\'afico; sobre todo en el per\'iodo post-guerra. Sin embargo, de manera reciente se ha empezado a mirar el crecimiento poblacional en \'Africa, Asia y Latinoam\'erica. Se desarroll\'o una ideolog\'ia particular llamada la ortodoxia demogr\'afica, que se basa en que el crecimiento poblacional siempre es indeseable en cualquier parte del mundo; mediante esta ideolog\'ia se convenci\'o el gobierno de EE UU que disminuir el crecimiento poblacional en pa\'ises en desarrollo era algo deseable y obtenible.\\
\\
Para los 70's,el inter\'es del p\'ublico en el crecimiento poblacional hab\'ia disminuido junto con las tasas de natalidad, por lo que surgi\'o un movimiento revisionista de la ortodoxia demogr\'afica, que alcanz\'o popularidad pol\'itica y social. Sobre este debate los demogr\'afos no se pronunciarion abiertamente, por lo cual, la tarea siguiente es mover el debate para poder corregir el exceso de legislaci\'on en pol\'iticas de poblaci\'on y desarrollar una teor\'ia cient\'ifica v\'alida acerca del crecimiento poblacional cuyas ideas centrales penetren los \'ambitos pol\'iticos, sociales y acad\'emicos.
\bibliographystyle{apalike}
\bibliography{bib_COLMEX}
\end{document}
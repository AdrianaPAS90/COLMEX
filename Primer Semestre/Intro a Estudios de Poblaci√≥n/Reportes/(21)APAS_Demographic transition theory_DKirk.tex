\documentclass[11pt,spanish,letterpaper]{article}
\usepackage[spanish,activeacute]{babel}
\usepackage{caption}
\usepackage{subcaption}
%\usepackage[latin1]{inputenc}
\usepackage{babel}
\usepackage{picture}
\usepackage{url}
\usepackage{mathrsfs}
\usepackage{amssymb,amsthm,amsmath,latexsym}
\usepackage[round]{natbib}
\usepackage{fancyhdr}
\theoremstyle{plain}
\newtheorem{teo}{Teorema}
\newtheorem{prop}[teo]{Proposicion}
\newtheorem{defi}[teo]{Definicion}
\newtheorem{obs}[teo]{Observacion}
\newtheorem{lem}[teo]{Lema}
\newtheorem{cor}[teo]{Corolario}
\usepackage[pdftex]{color,graphicx}
\newcommand{\sgn}{\mathop{\mathrm{sgn}}}
\setlength{\textwidth}{12.6cm}
\setlength{\textheight}{19cm}
\begin{document}
\begin{flushleft}
Adriana P\'erez-Arciniega Sober\'on
\end{flushleft}
\begin{center}
\textbf{Demographic transition theory}
\end{center}
En el art\'iculo de \cite{kirk1996demographic} se establece que, en resumen, la teor\'ia de la transici\'on demogr\'afica indicar\'ia que para las sociedades que se van modernizando, progresan de un estado pre-moderno de alta fecundidad y alta mortalidad a uno post-moderno en donde tanto la mortalidad y la fecundidad son bajas.\\
\\
La teor\'ia de la transici\'on demogr\'afica se empez\'o a construir a principios del siglo XX, cuando varios dem\'ografos detectaron que en las distintas poblaciones se pod\'ian observar distntos comportamientos demogr\'aficos y se delinearon tres categor\'ias, una donde las tasas de crecimiento disminu\'ian, otra donde tanto la mortalidad como la fecundidad hab\'ian disminuido pero segu\'ia existiendo diferencias entre estos decrecimientos y, por \'ultimo, donde ni la mortalidad ni la natalidad hab\'ian disminuido. La teor\'ia de la transici\'on dmeogr\'afica se cimenta como abstracci\'on de resultados de estudios de 1994.\\
\\
Una de las mayores cr\'iticas a este modelo se centraba acerca de que esta teor\'ia generalizaba la disminuci\'on en tasas de fecundidad sin tomar en cuenta las diferencias culturales y en la falta de precisi\'on al calcular el umbral requerido para la baja de fecundidad. Al respecto, en 1963 se organiz\'o el Proyecto de Fecundidad Europeo, que se dedic\'o a analizar la disminuci\'on de la fecundidad en el siglo XIX y se encontr\'o que la disminuic\'on de fecundidad ocurri\'o bajo distintas condiciones econ\'omicas y sociales, que una vez que la planificaci\'on familiar se hab\'ia puesto en marcha era irreversible y que son los entornos culturales los que influyen en la disminuci\'on de fecundidad. Igualmente, al tratar de rastrear la disminuci\'on de la mortalidad se observ\'o que esta dependi\'o en gran parte en el desarrollo de la modernidad mas que en nivel de ingreso y que aunque empez\'o a finales del siglo XVII y principios del siglo XIX, con la revoluci\'on farmace\'utica y m\'edica en el siglo XX  esta termin\'o de disminuir definitivamente.\\
\\
En la b\'usqueda de causas sobre este proceso de transici\'on demogr\'afica se observa tambi\'en la teor\'ia econ\'omica, dado que con la modernizaci\'on se ve a los costos de crianza de los hijos como un argumento econ\'omico de baja en la fecundidad. Otra de las causas estudiadas fue las diferencias culturales e ideol\'ogicas derivadas de la modernizaci\'on, como la secularizaci\'on de la vida e individualismo donde se pod\'ia ver a los hijos como un obst\'aculo a la realizaci\'on personal derivando en una baja en la fecundidad. Por supuesto, otro factor causal es la emancipaci\'on de las mujeres en la esfera dom\'estica y el acuerdo entre la pareja entre una cantidad menor de hijos aunado al rol gubernamental al implementar pol\'iticas p\'ublicas que incentivaban una menor cantidad de hijos y bajas en la mortalidad.\\
\\
Actualmente, las tasas de mortalidad han disminuido en todo el mundo y en los pa\'ises m\'as industrializados se cree que ya han alcanzado el fin de la transici\'on de mortalidad. La transici\'on de fecundidad igualmente est\'a a punto de volverse universal, aunque en los pa\'ises en v\'ias de desarrollo la disminuci\'on a desacelerado el ritmo comparado con el de la d\'ecada de 1980 y en algunos pa\'ises europeos la tasa de natalidad es menor al nivel de reemplazo, por lo que se compensan.\\
\\
El valor actual de la teor\'ia de la transici\'on demogr\'afica radica en el hecho de que no existe una mejor teor\'ia para pronosticar tendencias poblacionales futuras pero su status sigue siendo debatido, al igual que su papel institucional y sus bases filos\'oficas.
\bibliographystyle{apalike}
\bibliography{bib_COLMEX}
\end{document}
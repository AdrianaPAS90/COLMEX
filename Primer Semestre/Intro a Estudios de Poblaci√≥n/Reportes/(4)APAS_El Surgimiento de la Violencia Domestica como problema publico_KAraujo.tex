\documentclass[10pt,spanish,letterpaper]{article}
\usepackage[spanish,activeacute]{babel}
\usepackage{caption}
\usepackage{subcaption}
%\usepackage[latin1]{inputenc}
\usepackage{babel}
\usepackage{picture}
\usepackage{url}
\usepackage{mathrsfs}
\usepackage{amssymb,amsthm,amsmath,latexsym}
\usepackage[round]{natbib}
\usepackage{fancyhdr}
\theoremstyle{plain}
\newtheorem{teo}{Teorema}
\newtheorem{prop}[teo]{Proposicion}
\newtheorem{defi}[teo]{Definicion}
\newtheorem{obs}[teo]{Observacion}
\newtheorem{lem}[teo]{Lema}
\newtheorem{cor}[teo]{Corolario}
\usepackage[pdftex]{color,graphicx}
\newcommand{\sgn}{\mathop{\mathrm{sgn}}}
\setlength{\textwidth}{12.6cm}
\setlength{\textheight}{19cm}
\begin{document}
\begin{flushleft}
Adriana P\'erez-Arciniega Sober\'on
\end{flushleft}
\begin{center}
\textbf{El surgimiento de la violencia dom\'estica como problema p\'ublico y objeto de pol\'iticas}
\end{center}
\cite{araujo2000surgimiento} busca analizar c\'omo la violencia dom\'estica se volvi\'o p\'ublico de inter\'es gubernamental y social. Las pol\'iticas se establecen cuando la experiencia de ciertos actores se vuelve indeseable y mediante l\'ogica pol\'itico-institucionales, se dise$\tilde{n}$an normativas para prevenirlas y se motiva una acci\'on colectiva para que el suceso se vuelva inaceptable.\\
\\
Se dice que un tema no se vuelve un movimiento social hasta que se platica y se forma comunidad alrededor de ello. El tema de la violencia dom\'estica en Chile se comenz\'o a gestar a finales de los setenta, mediante la conformaci\'on de ONG's y organizaciones populares que, en su momento, fueron creadas para restaurar el tejido social desgarrado por la violencia pol\'itica y la resistencia al r\'egimen militar. Sin embargo, las mujeres que participaban en estas organizaciones poco a poco comenzaron a compartir sus experiencias de violencia dom\'estica, normalmente impartida por su c\'ontuge, independientemente de sus afiliaciones pol\'iticas. Gracias a estas organizaciones, se empieza a nombrar el problema de la violencia como producto de las relaciones de poder entre las clases sociales y g\'eneros.\\
\\
Ya en los a$\tilde{n}$os ochenta, la violencia dom\'estica deja de ser un problema construido y nombrado por mujeres para pasar a la esfera p\'ublica y alcanzar una preocupaci\'on colectiva. Se formula un nuevo marco interpretativo sobre la discriminaci\'on y violencia a la mujer en un enfoque estructural m\'as que como casos aislados de violencia f\'isica contra algunas mujeres. En un principio se ve\'ia a la violencia dom\'estica como la m\'axima representaci\'on de las relaciones desiguales de poder entre hombres y mujeres, que pod\'ian solucionarse mediante grupos de autoayuda y superaci\'on que 'empoderaran' a la mujer, poni\'endola como responsable de la misma.\\
\\
Para finales de los ochenta, el problema de violencia dom\'estica se institucionaliz\'o con la creaci\'on de la Red Chilena contra la Violencia Dom\'estica y Sexual (REDCHVD), con lo que se demuestra que saberse afectada por el problema no necesariamente es suficiente para encontrar una soluci\'on y que se requiere de un apoyo m\'as especializado. Con el cambio en la concepci\'on de lo qu\'e significa violencia dom\'estica, se abre la oportunidad de incluir estrategias de soluci\'on en agendas pol\'iticas, promoci\'on de reformas legales, etc. La REDCHVD juega un papel fundamental en esta nueva \'optica institucional sobre la violencia dom\'estica como punto de articulaci\'on entre movimientos internacionales, ONG's nacionales y gobiernos locales en el \'animo de promover pol\'iticas.\\
\\
El art\'iculo establece que en menos de 25 a$\tilde{n}$os, el tema de la violencia dom\'estica paso de estar ausente en las agendas y en el debate p\'ublico a ser un problema social reconocido que urge ser sobrepasado. En este lapso, tambi\'en se requiri\'o la constituci\'on de las mujeres como sujetos sociales, capaces de organizarse para empujar temas en distintos \'ambitos pol\'iticos. Esto demuestra que la construcci\'on de los problemas p\'ublicos son construidos por diversos actores en distintos escenarios que confrontan discursos en distintos marcos interpretativos.
\bibliographystyle{apalike}
\bibliography{bib_COLMEX}
\end{document}
\documentclass[11pt,spanish,letterpaper]{article}
\usepackage[spanish,activeacute]{babel}
\usepackage{caption}
\usepackage{subcaption}
\usepackage[latin1]{inputenc}
\usepackage{babel}
\usepackage{picture}
\usepackage{url}
\usepackage{mathrsfs}
\usepackage{amssymb,amsthm,amsmath,latexsym}
\usepackage[round]{natbib}
\usepackage{fancyhdr}
\theoremstyle{plain}
\newtheorem{teo}{Teorema}
\newtheorem{prop}[teo]{Proposicion}
\newtheorem{defi}[teo]{Definicion}
\newtheorem{obs}[teo]{Observacion}
\newtheorem{lem}[teo]{Lema}
\newtheorem{cor}[teo]{Corolario}
\usepackage[pdftex]{color,graphicx}
\newcommand{\sgn}{\mathop{\mathrm{sgn}}}
%\author{Adriana Perez-Arciniega Soberon} 
\begin{document}
La demograf\'ia se puede definir como una disciplina que estudia las estructuras de las poblaciones
y de los procesos que las afectan. Se puede pensar que la demograf\'ia est\'a al servicio de las ciencias sociales; sin embargo, la demograf\'ia tambi\'en toma algunos conceptos de las ciencias sociales.\\
\\
Lo que distingue a la demograf\'ia es el rigor con el que se busca analizar la realidad. La demograf\'ia es lenta en elaborar teor\'ias, dado que prefieren un reflejo de la realidad m\'as preciso de lo que una teor\'ia sin sustento pudiera elaborar. La demograf\'ia es una disciplina que est\'a muy cercana al pulso de las nuevas ideas y los nuevos fen\'omenos que pudieran impactar estas estructuras de poblaci\'on, por lo cual, los intereses de investigaci\'on de la demograf\'ia son muy vastos y multidiscilinarios.
\end{document}
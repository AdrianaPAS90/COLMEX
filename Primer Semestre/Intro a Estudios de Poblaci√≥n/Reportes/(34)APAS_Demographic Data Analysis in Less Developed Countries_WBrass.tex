\documentclass[11pt,spanish,letterpaper]{article}
\usepackage[spanish,activeacute]{babel}
\usepackage{caption}
\usepackage{subcaption}
%\usepackage[latin1]{inputenc}
\usepackage{babel}
\usepackage{picture}
\usepackage{url}
\usepackage{mathrsfs}
\usepackage{amssymb,amsthm,amsmath,latexsym}
\usepackage[round]{natbib}
\usepackage{fancyhdr}
\theoremstyle{plain}
\newtheorem{teo}{Teorema}
\newtheorem{prop}[teo]{Proposicion}
\newtheorem{defi}[teo]{Definicion}
\newtheorem{obs}[teo]{Observacion}
\newtheorem{lem}[teo]{Lema}
\newtheorem{cor}[teo]{Corolario}
\usepackage[pdftex]{color,graphicx}
\newcommand{\sgn}{\mathop{\mathrm{sgn}}}
\setlength{\textwidth}{12.6cm}
\setlength{\textheight}{19cm}
\begin{document}
\begin{flushleft}
Adriana P\'erez-Arciniega Sober\'on
\end{flushleft}
\begin{center}
\textbf{Demographic Data Analysis in Less Developed Countries: 1946-1996}
\end{center}
El art\'iculo de \cite{brass1996demographic} relata como para principios de la d\'ecada de 1950 se reconoci\'o que no se pod\'ian analizar de la misma manera los datos demogr\'aficos de los pa\'ises desarrollados con buena calidad de informaci\'on, que los pa\'ises en v\'ias de desarrollo. Es por esto que se determin\'o que se deb\'ia utilizar nuevos m\'etodos, dado que es la naturaleza de los datos la que debe dictar los m\'etodos de an\'alisis y no al rev\'es. As\'i se desarrollaron m\'etodos indirectos para extraer informaci\'on del material existente, para poder contar con informaci\'on m\'as robusta y precisa. \\
\\
Esta informaci\'on se puede obtener mediante censos y, m\'as recientemente, mediante m\'etodos de muestreo como encuestas. Sin embargo, en los pa\'ises en v\'ias de desarrollo el problema reside en la falta de informaci\'on regular a lo largo del tiempo por lo que las t\'ecnicas de an\'alisis dependen de los censos  y se enfocan en los registros administrativos y las estad\'isticas vitales.\\
\\
Una de las cosas m\'as importantes de determinar en la sociedad de estudio es la distribuci\'on por edades y por sexo. Si consideramos que la migraci\'on en la poblaci\'on no es tan importante y excluimos del an\'alisis a los ni$\tilde{n}$os m\'as j\'ovenes, se puede escoger un modelo adecuado para calcular la distribuci\'on de tasas de edad espec\'ificas de fecundidad para calcular el n\'umero  estandarizado de nacimientos. Utilizando estos m\'etodos y mediante 'sobrevivencia en reversa' se pod\'ia estimar la mortalidad infatil y estimaciones m\'as precisas de la fecundidad. Otra de las t\'ecnicas m\'as usuales es reporte de nacimientos por terceros, as\'i se pueden construir tasas de fecundidad acumuladas a partir de nacimientos reportados. El ratio llamado 'mean parity' se refiere al n\'umero de hijos por mujer en un grupo de edad espec\'ifico para el momento de la encuesta o censo entre el n\'umero de mujeres en riesgo de tener hijos se denota como $P$. Cuando los niveles de fertilidad se mantienen constantes, los valores de $P$ en los distintos grupos de edad de mujeres son iguales a las tasas espec\'ificas por edad acumuladas por cohorte y se puede utilizar para estimar las tasas acttuales. Otra medida que cobra cada vez m\'as importancia son los ratios de paridad progresiva, que se refieren a la proporci\'on de mujeres con un n-\'esimo nacimiento y contin\'uan con (n+1)-\'esimo nacimiento, como indicador de cambio en la estructura familiar. \\
\\
Igualmente, se puede reportar la mortalidad se conf\'ia en los reportes de los familiares que le sobreviven, aunque estos m\'etodos se han refinado en los \'ultimos a$\tilde{n}$os. Estos f\'enomenos se clasifican por edad del sobreviviente, as\'i, las proporciones de difuntos y sobrevivientes da lugar a una serie de \'indices relacionados con la mortalidad a lo largo de la vida. La relaci\'on entre estos \'indices y tasas de mortalidad tradicionales es determinada a partir de tablas de vida modelos. A\'un con todos los incrementos en la calidad descritos, los m\'etodos con diferentes fortalezas y limitaciones se combinan para reducir los errores individuales dando lugar a los m\'etodos compuestos.\\
\\
Sin embargo, a\'un existen posibilidades de mejorar las t\'ecnicas de extracci\'on indicadores demogr\'aficos b\'asicos de informaci\'on inusual. Por ejemplo, las estimaciones de mortalidad basadas en las declaraciones de los sobrevivientes han tenido menos \'exito de lo que sugieren algunas teor\'ias. Con el aumento de temas relacionados a la mortalidad, como la mortalidad por SIDA que ha afectado la estructura de edades de este fen\'omenos, es necesaria la creaci\'on de mejores modelos de estimaci\'on y mejores t\'ecnicas de recolecci\'on de informaci\'on. 
\bibliographystyle{apalike}
\bibliography{bib_COLMEX}
\end{document}
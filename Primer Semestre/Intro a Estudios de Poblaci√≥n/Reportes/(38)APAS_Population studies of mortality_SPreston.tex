\documentclass[10pt,spanish,letterpaper]{article}
\usepackage[spanish,activeacute]{babel}
\usepackage{caption}
\usepackage{subcaption}
%\usepackage[latin1]{inputenc}
\usepackage{babel}
\usepackage{picture}
\usepackage{url}
\usepackage{mathrsfs}
\usepackage{amssymb,amsthm,amsmath,latexsym}
\usepackage[round]{natbib}
\usepackage{fancyhdr}
\theoremstyle{plain}
\newtheorem{teo}{Teorema}
\newtheorem{prop}[teo]{Proposicion}
\newtheorem{defi}[teo]{Definicion}
\newtheorem{obs}[teo]{Observacion}
\newtheorem{lem}[teo]{Lema}
\newtheorem{cor}[teo]{Corolario}
\usepackage[pdftex]{color,graphicx}
\newcommand{\sgn}{\mathop{\mathrm{sgn}}}
\setlength{\textwidth}{12.6cm}
\setlength{\textheight}{19cm}
\begin{document}
\begin{flushleft}
Adriana P\'erez-Arciniega Sober\'on
\end{flushleft}
\begin{center}
\textbf{Population studies of mortality}
\end{center}
En su art\'iculo \cite{preston1996population} hace una revisi\'on de los estudios de mortalidad en la revista \textit{Population Studies} durante la segunda mitad del siglo XX. Esta revisi\'on se divide en cuatro categor\'ias: medici\'on, niveles, tendencias y correlaciones a nivel individual.\\
\\
En la categor\'ia de mediciones, se toma como principio los indicadores derivados de la tabla de mortalidad. Aunque no se han realizados mejoras significativas en la construcci\'on misma de la tabla desde la Segunda Guerra Mundial, s\'i se ha habido una expansi\'on en la aplicaci\'on de m\'etricas de la tabla de mortalidad a datos no convencionales o en t\'ecnicas de contrucci\'on de la tabla para poblaciones que no contaban con datos suficientes. Uno de los mayores logros en la medici\'on de la mortalidad en el per\'iodo post-guerra, fue la t\'ecnica elaborada por William Brass para medir la proporci\'on de ni$\tilde{n}$os que hab\'ian muerto entre los que hab\'ian nacido, con la ayuda de solo una encuesta o censo. Otro logro en este per\'iodo se refiri\'o en los m\'etodos para estimar la mortalidad de adultos cuando los datos est\'an incompletos mediante la medici\'on de su grado de incompletitud.\\
\\
La mayor\'ia de estudios cualitativos que se dedicaron a estudiar la mortalidad, tuvieron oportunidad de realizar interpretaciones causales dado que la mortalidad se puede explicar dependiendo de los distintos factores subyacentes como el per\'iodo, grupo social, etc. para llegar a conclusiones sobre la historia de la poblaci\'on. La revista \textit{Population Studies} ha sido el mayor medio que publiza el an\'alisis de patrones de mortalidad y su progreso desde el origen de las estad\'isticas vitales.\\
\\
Si bien, la mayor\'ia de los art\'iculos referentes a mortalidad tratan de la medici\'on y descripci\'on de los niveles de mortalidad, ha habido un incremento en los estudios que tratan de identificar y explicar los factores asociados con las variaciones en los niveles de mortalidad. Por ejemplo, si los niveles de mortalidad tienen que ver con el ingreso, ingesta nutricional o a mejores t\'ecnicas para prevenir la muerte prematura.\\
\\
A ra\'iz de la publicaci\'on de la Encuesta Mundial de Fertilidad, se pudieron analizar datos a nivel individual a fines de los a$\tilde{n}$os 70's y 80's, lo cual dio oportunidada realizar an\'alisis indivudal. As\'i, se pudieron hacer mucho m\'as an\'alisis sobre la mortalidad infantil y las din\'amicas de fertilidad utilizando las nuevas t\'ecnicas de modelaje como la regresi\'on log\'istica y an\'alisis log-lineal que permitieron an\'alisis m\'as precisos del impacto de varias caracter\'isticas sobre el riesgo de morir. Por ejemplo, las diferencias socioecon\'omicas, la situaci\'on laboral de la madre o el lugar de residencia.\\
\\
Es probable que este clase de an\'alisis no se siga realizando; sin embargo, si proporcion\'o un progreso importante en el uso de informaci\'on mediante t\'ecnicas m\'as especializadas. En conclusi\'on, todos los art\'iculos que han estudiado la mortalidad en la revista \textit{Population Studies} han teorizado sobre c\'omo el comportamiento de la salud se relaciona con otras esferas del comportamiento y c\'omo mediante las oportunidades y limitaciones de la informaci\'on y de las t\'ecnicas de estudio, transmiten el efecto del pasado en el presente.
\bibliographystyle{apalike}
\bibliography{bib_COLMEX}
\end{document}
\documentclass[10pt,spanish,letterpaper]{article}
\usepackage[spanish,activeacute]{babel}
\usepackage{caption}
\usepackage{subcaption}
%\usepackage[latin1]{inputenc}
\usepackage{babel}
\usepackage{picture}
\usepackage{url}
\usepackage{mathrsfs}
\usepackage{amssymb,amsthm,amsmath,latexsym}
\usepackage[round]{natbib}
\usepackage{fancyhdr}
\theoremstyle{plain}
\newtheorem{teo}{Teorema}
\newtheorem{prop}[teo]{Proposicion}
\newtheorem{defi}[teo]{Definicion}
\newtheorem{obs}[teo]{Observacion}
\newtheorem{lem}[teo]{Lema}
\newtheorem{cor}[teo]{Corolario}
\usepackage[pdftex]{color,graphicx}
\newcommand{\sgn}{\mathop{\mathrm{sgn}}}
\setlength{\textwidth}{12.6cm}
\setlength{\textheight}{19cm}
\begin{document}
\begin{flushleft}
Adriana P\'erez-Arciniega Sober\'on
\end{flushleft}
\begin{center}
\textbf{The end of the demographic transition: relief or concern?}
\end{center}
\cite{vallin2002end} explica que en el siglo XIX existi\'o el miedo a la despoblaci\'on y a mediados del siglo XX se tem\'ia las consecuencias de la explosi\'on poblacional. Ambos miedos se concretizan en un solo proceso llamado transici\'on demogr\'afica, este paradigma permite entender los fen\'omenos demogr\'aficos actuales , poner miedos pasados en perspectiva y avanzar en la relaci\'on de proyecciones de poblaci\'on m\'as precisas. Actualmente, el reto m\'as grande de este paradigma es que aunque se plantea que el fin de la transici\'on para finales del siglo XXI, las consecuencias del mismo no paran al mismo tiempo.\\
\\
Los cambios demogr\'aficos se pueden rastrear a mediados del siglo XVII donde hab\'ia una alta fecundidad que contrarrestaba la alta mortalidad. Sin embargo, con la revoluci\'on industrial y los cambios que esta conllev\'o, en el per\'iodo de un siglo disminuy\'o la cantidad de hijos por mujer de seis al necesario para el reemplazo poblacional, de dos hijos por mujer. La mortalidad tambi\'en disminuy\'o de manera significativa, debido a los avances tecnol\'ogicos y m\'edicos; la disminuci\'on en mortalidad sucedi\'o de manera mucho m\'as acelerada que la correspondiente a la fecundidad, por lo que esto tambi\'en provoc\'o un cambio en la estructura de edades de la poblaci\'on.\\
\\
En 1980 la ONU proyect\'o una estabilizaci\'on de la poblaci\'on de 9.5 mil millones de personas para 2050 y 11 mil millones para finales del siglo XXI. Si en 2000 ya se contaba con una poblaci\'on de 6 mil millones, se podr\'ia pensar que el fin de la transici\'on dmeogr\'afica est\'a cerca pero se deben tomar en cuenta dos puntos. El primero, es que el 20\% de la poblaci\'on mundial posee el 80\% del ingreso mundial y el segundo, que la mayor\'ia del crecimiento poblacional est\'a sucediendo en pa\'ises en v\'ias de desarrollo, por lo que los encargados de este crecimiento son los pa\'ises m\'as pobres; estos dos puntos denotan desigualdades econ\'omicas y sociales que no coresponden al paradigma de una poblaci\'on estabilizada.\\
\\
Se puede entender el fin de la transici\'on demogr\'afica de maneras distintas. Se puede pensar que el fin de la transici\'on demogr\'afica es cuando la mortalidad y la fecundidad terminan de decrecer y se estabilizan, pero se debe considerar al cambio en la estructura de edad de la poblaci\'on como un factor determinante en la velocidad del cambio demogr\'afico. Por el lado de la fecundidad, si cada vez se tienen menos hijos, la base de la pir\'amide poblacional se ir\'a encogiendo, la proporci\'on de poblaci\'on joven ir\'a disminuyendo mientras que la de adultos mayores incrementa. Por el lado de la mortalidad, si la fecundidad disminuye el decremento en mortalidad contrarresta el envejecimiento poblacional desde abajo , pero si las tasas de mortalidad infantil disminuyen incrementa la poblaci\'on mayor, por lo que se dice que la poblaci\'on envejece desde arriba. Es este proceso de envejecimiento poblacional lo que da lugar a la teor\'ia de las poblaciones estables, esto quiere decir poblaciones con todos los par\'ametros constantes y la estructura de edades determinada por la curva de sobrevivencia.\\% As\'i, en una poblaci\'on con esperanza de vida de 85 a$\tilde{n}$os y con una tasa de natalidad de 2.1 hijos por mujer, se tiene una poblaci\'on donde el 24\% de la poblaci\'on es menor a 20 a$\tilde{n}$os, 46\% de la poblaci\'on tiene entre 20 y 59 a$\tilde{n}$os y 30\% en edades mayores a 60.\\
\\
Aunque a principios de la d\'ecada de 1980, la ONU se dedic\'o a preparar planes para la estabilizaci\'on de la poblaci\'on, esta no se ha podido lograr debido a varios factores. Uno de estos fatores es la disparidad en el decrecimiento de la fecundidad. Este decrecimiento puede deberse a dos razones: la primera es que las mujeres est\'an decidiendo tener solo un hijo o que no se ha modificado la cantidad de hijos deseados (2.1), pero que se est\'a posponiendo el momento de tenerlos. Suponiendo el primer escenario, siguiendo los procesos de envejecimiento normales, la poblaci\'on mundial disminuir\'ia en tama$\tilde{n}$o de manera indefinida hasta desaparecer. En el segundo escenario, la poblaci\'on no desaparecer\'ia y la poblaci\'on llegar\'ia a la estructura de edad estable, pero el proceso ser\'ia largo dado que se alternar\'ian cohortes muy peque$\tilde{n}$as con unas muy grandes.\\
\\
Otro de los grandes factores que influyen en la teor\'ias de las poblaciones estables es la esperanza de vida, sobre esto tambi\'en se tienen dos perspectivas. La primera, donde se plantea que ya se est\'a muy cerca de empatar con los l\'imites humanos de longevidad; es decir, que se ha logrado acercar la media (esperanza de vida) y el m\'aximo (longevidad). En la segunda perspectiva, la longevidad humana es susceptible a cambios y va aumentando con el tiempo, de acuerdo a las condiciones de vida, sociales y m\'edicas.\\
\\
Por \'ultimo, el autor exhorta a pensar en maneras de asegurar el desarrollo de las regiones m\'as empobrecidas del mundo, dado que ser\'an las que absorberan la mayor parte del crecimiento poblacional futuro. Tambi\'en, en prestar atenci\'on a los roles, expectativas y necesidades de todos los grupos de edad para atender la estructura de edad modificada. Finalmente, a dise$\tilde{n}$ar pol\'iticas p\'ublicas que no solo busquen acelerar el decrecimiento en la fecundidad sino a estabilizar poblaciones incentivando la fecundidad donde sea demasiado baja y, mientros esto sucede, aprovecharse de los flujos migratorios.
\bibliographystyle{apalike}
\bibliography{bib_COLMEX}
\end{document}
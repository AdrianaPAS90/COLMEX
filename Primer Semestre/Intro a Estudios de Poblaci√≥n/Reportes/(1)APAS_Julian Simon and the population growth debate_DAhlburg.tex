\documentclass[11pt,spanish,letterpaper]{article}
\usepackage[spanish,activeacute]{babel}
\usepackage{caption}
\usepackage{subcaption}
%\usepackage[latin1]{inputenc}
\usepackage{babel}
\usepackage{picture}
\usepackage{url}
\usepackage{mathrsfs}
\usepackage{amssymb,amsthm,amsmath,latexsym}
\usepackage[round]{natbib}
\usepackage{fancyhdr}
\theoremstyle{plain}
\newtheorem{teo}{Teorema}
\newtheorem{prop}[teo]{Proposicion}
\newtheorem{defi}[teo]{Definicion}
\newtheorem{obs}[teo]{Observacion}
\newtheorem{lem}[teo]{Lema}
\newtheorem{cor}[teo]{Corolario}
\usepackage[pdftex]{color,graphicx}
\newcommand{\sgn}{\mathop{\mathrm{sgn}}}
\setlength{\textwidth}{12.6cm}
\setlength{\textheight}{19cm}
\begin{document}
\begin{flushleft}
Adriana P\'erez-Arciniega Sober\'on
\end{flushleft}
\begin{center}
\textbf{Julian Simon and the population growth debate}
\end{center}
\cite{ahlburg1998julian} habla de la importancia que tuvo el economista Julian Simon en el an\'alisis de la relaci\'on entre poblaci\'on, desarrollo y pol\'iticas p\'ublicas, dado que \'el cre\'ia que el paradigma de que el crecimiento poblacional era algo malo, era err\'oneo. El prop\'osito de este art\'iculo es discutir sus aportaciones al debate del crecimiento poblacional a lo largo de su carrera.\\
\\
Julian Simon entr\'o a la discusi\'on sobre crecimiento poblacional a mediados de los setentas, en un momento caracterizado, de acuerdo a Dennis Hodgson y Susan Watkins (1997), por "una degradaci\'on significativa del control de fecundidad en la agenda internacional de intervenciones pol\'iticas necesarias" y debati\'o ferozmente a la ideolog\'ia neomalthusiana. En uno de sus primeros trabajos, Julian Simon argumentaba que aunque los efectos a corto plazo del crecimiento poblacional fueran negativos, a largo plazo estos ser\'ian en gran parte positivos. Aunque fue calificado de optimista, Simon prob\'o, mediante un modelo econ\'omico que sus alegatos estaban fundamentados; pero tambi\'en especific\'o que en pa\'ises en v\'ias de desarrollo los efectos negativos podr\'ian ser m\'as graves y que los efectos positivos podr\'ian tardar un poco m\'as en hacerse presentes. \\
\\
Sin embargo, mediante nuevas revisiones con la tecnolog\'ia actual de simulaciones podr\'ia aseverar que los argumentos de Simon s\'i fueron demasiado optimistas y que aunque fueron suficientes para causar desconfianza entre los malthusianos, no son lo suficientemente fundamentadas para rechazar el punto de vista alarmista de manera definitiva. Muchos estudios actuales encuentran asociasiones negativas entre el crecimiento poblacional y el crecimiento del ingreso per c\'apita, contradiciendo a Simon y que los efectos positivos de los cuales hablaba no son suficientes a largo plazo para contrarrestar los efectos negativos.\\
\\
El componente principal de los argumentos positivos de Julian Simon, son las personas. Cre\'ia que una mayor poblaci\'on daba oportunidad a que se produjera m\'as conocimiento y mentes m\'as entrenadas para hacer frente a cualquier reto que el crecimiento poblacional presentara. Uno de los ejemplos m\'as recientes podr\'ia ser el desarrollo de tecnolog\'ia agr\'icola, abriendo la posibilidad de poder satisfacer cualquier demanda alimenticia futura. Aun as\'i, surgen contra argumentos que recomiendan precauci\'on; por ejemplo, con la teconolog\'ia agr\'icola, se han estudiado los rendimientos de los campesinos y se ha podido concluir que la expansi\'on de riesgo para los agricultores es cada vez m\'as costosa, por lo que su crecimiento no es sostenible.\\
\\
De este modo, \cite{ahlburg1998julian} concluy\'e que aunque los estudios de Julian Simon no fueron definitorios en cambiar el paradigma sobro el crecimiento poblacional, s\'i fue un detonante importante para obligar al campo a pensar de manera distinta y explorar otras corrientes de pensamiento.
\bibliographystyle{apalike}
\bibliography{bib_COLMEX}
\end{document}
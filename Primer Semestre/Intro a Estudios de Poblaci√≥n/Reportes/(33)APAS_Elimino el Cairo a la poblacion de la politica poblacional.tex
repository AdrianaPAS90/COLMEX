\documentclass[11pt,spanish,letterpaper]{article}
\usepackage[spanish,activeacute]{babel}
\usepackage{caption}
\usepackage{subcaption}
%\usepackage[latin1]{inputenc}
\usepackage{babel}
\usepackage{picture}
\usepackage{url}
\usepackage{mathrsfs}
\usepackage{amssymb,amsthm,amsmath,latexsym}
\usepackage[round]{natbib}
\usepackage{fancyhdr}
\theoremstyle{plain}
\newtheorem{teo}{Teorema}
\newtheorem{prop}[teo]{Proposicion}
\newtheorem{defi}[teo]{Definicion}
\newtheorem{obs}[teo]{Observacion}
\newtheorem{lem}[teo]{Lema}
\newtheorem{cor}[teo]{Corolario}
\usepackage[pdftex]{color,graphicx}
\newcommand{\sgn}{\mathop{\mathrm{sgn}}}
\setlength{\textwidth}{12.6cm}
\setlength{\textheight}{19cm}
\begin{document}
\begin{flushleft}
Adriana P\'erez-Arciniega Sober\'on
\end{flushleft}
\begin{center}
\textbf{?`Elimin\'o El Cairo a la poblaci\'on de las pol\'iticas poblacionales?}
\end{center}
En el art\'iculo de \cite{murphy1997elimino} se exploran las conclusiones derivadas de la Conferencia Internacional de Poblaci\'on y Desarrollo realizada en 1994 en El Cairo. Una de las grandes conclusiones es que esta conferencia llev\'o la conversaci\'on sobre poblaci\'on de un nivel macro a uno micro, dejando de lado a las poblaciones pra concentrarse en la vida y acciones individuales. Este cambio de enfoque, aunque aceptado por las agencias globales y las donadoras externas, caus\'o recelo por abandonar la idea del crecimiento poblacional. Sin embargo, \cite{murphy1997elimino} argumentan que este nuevo enfoque resulta m\'as integral en los temas de salud reproductiva.\\
\\
Previo a la conferencia, organizaciones feministas y de salud argumentaban que el bienestar de las mujeres se hab\'ia descuidado por las agencias de financiamiento, en aras de enfocarse m\'as en el problema del crecimento poblacional y su impacto ambiental. De acuerdo con el enfoque revisionista, las acciones gubernamentales para planificaci\'on familiar deben considerarse en el contexto educativo y cultural espe c\'ifico a cada sociedad. La intervenci\'on gubernamental en la salud reproductiva se justifica mediante el argumento de justicia social, que solicita que el gobierno establezca una infraestructura b\'asica de salud para hacer llegar m\'etodos de planificaci\'on familiar a toda la poblaci\'on.\\
\\
El Banco Mundial ayud\'o al financiamiento de estas infraestructuras mediante el desarrollo de un m\'etodo llamado BOD, para evaluar las inversiones de salud en distintos pa\'ises. En pa\'ises con alta mortalidad infantil y alta fecundidad, el m\'etodo BOD asigna prioridad alta a las intervenciones en salud reproductiva; de igual manera, cuando sucede lo contrario, se asigna una baja prioridad. Los problemas que surgen mediante este m\'etodo, es que las complicaciones durante el embarazo no son registradas, lo cual afecta el puntaje para asignar prioridad alta o baja de intervenci\'on en salud reproductiva. Otra limitaci\'on, es que el m\'etodo BOD no considera las implicaciones culturales de mejores pr\'acticas de salud reproductiva.\\
\\
Las conclusiones de El Cairo indican que el \'enfasis en salud reproductiva no debe ser supeditada solamente a indicadores, sino centrarse en el bienestar del individuo.
\bibliographystyle{apalike}
\bibliography{bib_COLMEX}
\end{document}
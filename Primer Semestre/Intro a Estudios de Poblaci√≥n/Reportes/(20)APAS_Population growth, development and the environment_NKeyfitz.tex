\documentclass[11pt,spanish,letterpaper]{article}
\usepackage[spanish,activeacute]{babel}
\usepackage{caption}
\usepackage{subcaption}
%\usepackage[latin1]{inputenc}
\usepackage{babel}
\usepackage{picture}
\usepackage{url}
\usepackage{mathrsfs}
\usepackage{amssymb,amsthm,amsmath,latexsym}
\usepackage[round]{natbib}
\usepackage{fancyhdr}
\theoremstyle{plain}
\newtheorem{teo}{Teorema}
\newtheorem{prop}[teo]{Proposicion}
\newtheorem{defi}[teo]{Definicion}
\newtheorem{obs}[teo]{Observacion}
\newtheorem{lem}[teo]{Lema}
\newtheorem{cor}[teo]{Corolario}
\usepackage[pdftex]{color,graphicx}
\newcommand{\sgn}{\mathop{\mathrm{sgn}}}
\setlength{\textwidth}{12.6cm}
\setlength{\textheight}{19cm}
\begin{document}
\begin{flushleft}
Adriana P\'erez-Arciniega Sober\'on
\end{flushleft}
\begin{center}
\textbf{Population growth, Development and the Environment}
\end{center}
El art\'iculo de \cite{keyfitz1996population} habla sobre la relaci\'on que existe entre desarrollo y ambiente y c\'omo esta relaci\'on deber\'ia ser de vital inter\'es para los dem\'ografos. Las preocupaciones en el campo han ido cambiando con el tiempo, aunque al principio se preocupaba de la selecci\'on gen\'etica, la pregunta sobre la extinci\'on natural de las poblaciones originales europues y la calidad de la poblaci\'on. Despu\'es, la preocupaci\'on fue el r\'apido crecimiento poblacional y la transici\'on demogr\'afica que se encargo de ello. Con el crecimiento poblacional surgen nuevas preocupaciones \'eticas de control de la poblaci\'on en los pa\'ises menos desarrollados y el auge de ideolog\'ias nacionalista con la presunta extinci\'on de alguna raza.\\
\\
Como remedio a esta extinci\'on, el art\'iculo cita la migraci\'on internacional. Aunque este fen\'omeno empez\'o en la post-guerra, permaneci\'o el resto del siglo. Este crecimiento poblacional oblig\'o a los gobiernos y a las econom\'ias a pensar en las condiciones que deb\'ian permanecer y ser heredadas a generaciones venideras. Dado que la demograf\'ia se ha quedado corta en el c\'alculo de poblaciones, se insta a una \'optica interdisciplinaria para el estudio de estos fen\'omenos. Es importante distinguir en la capacidad de an\'alisis entre las distintas disciplinas, pues lo que un bi\'ogo o economista estudia a corto plazo, el dem\'ografo analiza a d\'ecadas.\\
\\
Ese es el problema con las repercusiones ambientales, que se retrasan d\'ecadas suficientes para que la econom\'ia se acelere y la poblaci\'on en crecimiento tengan efectos negativos sobre uno o m\'as elementos ambientales. Y este deterioro ambiental no puede suceder sin tener efectos negativos en la econom\'ia. Este c\'irculo, podr\'ia llevar al colapso de la poblaci\'on.\\
\\
El problema con los recursos naturales es que han sido demasiado explotados, pues deben servir a una poblaci\'on espec\'ifica. Los impactos ambientales no se deben del todo a la riqueza occidental, sino a la forma en que se deciden utilizar los recursos naturales; esta forma es determinada por la cultura. Al final, lo que se concluye es que de persistir en estos patrones de consumo de los recursos naturales, se vivir\'a a expensas de generaciones futuras. Es por eso que urge el dise$\tilde{n}$o de patrones de consumo m\'as amigables con el medio ambiente y la estabilizaci\'on de la poblaci\'on. 
\bibliographystyle{apalike}
\bibliography{bib_COLMEX}
\end{document}
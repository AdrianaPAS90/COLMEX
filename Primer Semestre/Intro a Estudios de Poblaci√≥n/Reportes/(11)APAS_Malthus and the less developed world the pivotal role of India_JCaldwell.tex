 \documentclass[11pt,spanish,letterpaper]{article}
\usepackage[spanish,activeacute]{babel}
\usepackage{caption}
\usepackage{subcaption}
%\usepackage[latin1]{inputenc}
\usepackage{babel}
\usepackage{picture}
\usepackage{url}
\usepackage{mathrsfs}
\usepackage{amssymb,amsthm,amsmath,latexsym}
\usepackage[round]{natbib}
\usepackage{fancyhdr}
\theoremstyle{plain}
\newtheorem{teo}{Teorema}
\newtheorem{prop}[teo]{Proposicion}
\newtheorem{defi}[teo]{Definicion}
\newtheorem{obs}[teo]{Observacion}
\newtheorem{lem}[teo]{Lema}
\newtheorem{cor}[teo]{Corolario}
\usepackage[pdftex]{color,graphicx}
\newcommand{\sgn}{\mathop{\mathrm{sgn}}}
\setlength{\textwidth}{12.6cm}
\setlength{\textheight}{19cm}
\begin{document}
\begin{flushleft}
Adriana P\'erez-Arciniega Sober\'on
\end{flushleft}
\begin{center}
\textbf{Malthus and the less developed world: the pivotal role of India}
\end{center}
El art\'iculo de \cite{caldwell1998malthus} trata de explicar por qu\'e las ideas de Malthus sobre el crecimiento poblacional siguen vigentes despu\'es de 200 a$\tilde{n}$os y c\'omo estas ideas reflejan la relaci\'on entre Inglaterra e India.\\
\\
Cuando el ensayo de Malthus fue escrito, la econom\'ia inglesa se reg\'ia bajo los principios econ\'omicos de Adam Smith aunque con el doble de poblaci\'on y el Estado ten\'ia implementadas leyes y programas ocupados en el alivio de la pobreza. As\'i Malthus describi\'o un sistema que pod\'ia autorregularse y, a la vez, explicar las fluctuaciones poblacionales. De este modo, el sistema dictaba que la fecundidad humana sobrepasara la producci\'on de alimentos y, cuando esto hab\'ia sucedido, la naturaleza lo regulaba mediante epidemias o guerras. Mientras que en la actualidad malthusiana, mediante los programas de apoyo a la pobreza, la poblaci\'on hab\'ia afectado su propiedad de autorregulaci\'on y segu\'ia creciendo. Estas ideas fueron r\'apidamente aceptadas por la clase alta, mientras que fue rechazada por la clase m\'as empobrecida, debido a que implicaba el cese de los apoyos que estaban recibiendo. Sin embargo, una vez que estas ideas fueron aceptadas por los intelectuales ingleses d la \'epoca, se empezaron a formular programas y acciones que promovieran la anticoncepci\'on entre la clase trabajadora.\\
\\
Cuando se empezaron a mirar poblaciones agrarias bajo la \'optima malthusiana, se pens\'o primero en colobias inglesas como los nativos americanos o los abor\'igenes australianos, no fue sino hasta generaciones posteriores que identificaron crisis malthusianas en lugares como China o India, es decir, sociedades agrarias densamente pobladas. Los que administraban la colonia india adoptaron varias ideas malthusianas como que la guerra, la enfermedad y las enfermedades eran instrumentos de control poblacional o tambi\'en que las intervenciones en crisis de hambrunas podr\'ian ser m\'as perjudiciales que beneficiosas.\'
\\

La colonizaci\'on inglesa no solo fue de manra f\'isica sino tambi\'en intelectual. La \'elite india tambi\'an hablaba ingl\'es y se iba a educar a Inglaterra, de esta manera el enfoque malthusiano no fue solo adoptado por las autoridades coloniales, sino tambi\'en por los gobernantes nativos. Esto fue instrumental para la fijaci\'on de pol\'iticas de control poblacional en el territorio indio. 
\bibliographystyle{apalike}
\bibliography{bib_COLMEX}
\end{document}
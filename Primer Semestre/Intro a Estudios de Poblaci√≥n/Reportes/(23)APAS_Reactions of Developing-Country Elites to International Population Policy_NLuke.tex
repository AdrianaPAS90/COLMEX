\documentclass[10pt,spanish,letterpaper]{article}
\usepackage[spanish,activeacute]{babel}
\usepackage{caption}
\usepackage{subcaption}
%\usepackage[latin1]{inputenc}
\usepackage{babel}
\usepackage{picture}
\usepackage{url}
\usepackage{mathrsfs}
\usepackage{amssymb,amsthm,amsmath,latexsym}
\usepackage[round]{natbib}
\usepackage{fancyhdr}
\theoremstyle{plain}
\newtheorem{teo}{Teorema}
\newtheorem{prop}[teo]{Proposicion}
\newtheorem{defi}[teo]{Definicion}
\newtheorem{obs}[teo]{Observacion}
\newtheorem{lem}[teo]{Lema}
\newtheorem{cor}[teo]{Corolario}
\usepackage[pdftex]{color,graphicx}
\newcommand{\sgn}{\mathop{\mathrm{sgn}}}
\setlength{\textwidth}{12.6cm}
\setlength{\textheight}{19cm}
\begin{document}
\begin{flushleft}
Adriana P\'erez-Arciniega Sober\'on
\end{flushleft}
\begin{center}
\textbf{Reactions of Developing-Country Elites to International Population Policy}
\end{center}
El art\'iculo \cite{luke2002reactions} habla de c\'omo fue difundida y aceptada la nueva pol\'itica poblacional en los p\'ises en desarrollo acordada en la Conferencia Internacional de Poblaci\'on y Desarrollo en El Cairo en 1994. Estas reacciones fueron obetenidas mediante entrevistas a las \'elites gubernamentales y de tomadores de decisiones en el \'ambito de salud en los pa\'ises en desarrollo. El an\'alisis se realiza mediante dos ennfoques: el consenso normativo, que se basa en la idea que la atracci\'on en el nuevo dogma cultural llevar\'a a la formaci\'on de un consenso que soporte su dispersi\'on; y los diferenciales de poder y donadores externos, en el cual se enfatiza las diferencias de poder entre los miembros de una comunidad global, particularmente entre los grupos que promueven el nuevo dogma y los grupos hacia los cuales est\'a orientado.\\
\\
En El Cairo se enfoc\'o la conversaci\'on m\'as en salud reproductiva (planeaci\'on familiar, salud de la madre y ETS) y equidad de g\'enero (econ\'omica y legal) que en el crecimiento poblacional o el enfoque neo-malthusiano, que era el enfoque reinante en las pol\'iticas de poblaci\'on internacionales para principios de los 90's. A pesar de que en la conferencia, todos los pa\'ises participantes firmaron el acuerdo de las nuevas pol\'iticas, no pareci\'ian estar muy entusiasmados con la implementaci\'on local de las mismas.\\
\\
Uno de los primeros indicadores del \'exito de un nuevo dogma es el entusiasmo discursivo o en la ret\'orica, dado la importancia del lenguaje en las culturas. Hubo pa\'ises que pensaron que la ret\'orica utilizada estaba demasiado occidentalizada para sus valores tomando con precauci\'on el programa y hubo otros pa\'ises se entusiasmaron con el discurso, trataron de superar las diferencias culturales y hacer cambios en las acciones locales. Se hizo evidente que la mayor\'ia de los pa\'ises se dedicaron a escoger solo las partes del programa que les serv\'ian y que las \'elites pensaban que eran relevantes y desechando lo dem\'as, alegando las extremas diferencias culturales. Los programas m\'as populares fueron los de planeaci\'on familiar y maternidad segura, que eran los programas que se estaban promoviendo anteriormente y programas como los de equidad de g\'enero no recibieron apoyo.\\
\\
Igualmente, una de las mayores razones por la cual las \'elites de los pa\'ises en desarrollo decidieron no llevar a cabo el programa en su totalidad fue por falta de financiamiento. El financiamiento que s\'i se recib\'ia se destinaba a programas de m\'as alto perfil, definido por los donadores externos, poniendo en duda la soberan\'ia del pa\'is al definir por si mismo las prioridades de salud. La falta de financiamiento demostr\'o que la mayor\'ia de los pa\'ises en desarrollo apenas ten\'ian capacidad para sobrellevar los temas m\'inimo como para poder expandir su agenda por s\'i mismos y que, las donadoras externas tienen su propia agenda.\\
\\
Aunque en El Cairo parec\'ia que los valores occidentales de equidad de g\'enero y salud reproductiva hab\'ian triunfado, qued\'o en evidencia que hay una diferencia entre formular pol\'iticas p\'ublicas e implementar una pol\'itica local. Por lo que las reacciones ambivalentes a El Cairo pueden ser interpretadas como un ejemplo de difusi\'on cultural espont\'anea y consenso normativo o un ejemplo de difusi\'on cultural basada en el realismo y diferenciamiento de los recursos.
\bibliographystyle{apalike}
\bibliography{bib_COLMEX}
\end{document}
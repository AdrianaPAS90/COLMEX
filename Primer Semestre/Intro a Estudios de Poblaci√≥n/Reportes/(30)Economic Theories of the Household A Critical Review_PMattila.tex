\documentclass[11pt,spanish,letterpaper]{article}
\usepackage[spanish,activeacute]{babel}
\usepackage{caption}
\usepackage{subcaption}
%\usepackage[latin1]{inputenc}
\usepackage{babel}
\usepackage{picture}
\usepackage{url}
\usepackage{mathrsfs}
\usepackage{amssymb,amsthm,amsmath,latexsym}
\usepackage[round]{natbib}
\usepackage{fancyhdr}
\theoremstyle{plain}
\newtheorem{teo}{Teorema}
\newtheorem{prop}[teo]{Proposicion}
\newtheorem{defi}[teo]{Definicion}
\newtheorem{obs}[teo]{Observacion}
\newtheorem{lem}[teo]{Lema}
\newtheorem{cor}[teo]{Corolario}
\usepackage[pdftex]{color,graphicx}
\newcommand{\sgn}{\mathop{\mathrm{sgn}}}
\setlength{\textwidth}{12.6cm}
\setlength{\textheight}{19cm}
\begin{document}
\begin{flushleft}
Adriana P\'erez-Arciniega Sober\'on
\end{flushleft}
\begin{center}
\textbf{Economic Theories of the Household: A Critical Review}
\end{center}
\cite{mattila1999working} trata de explicar la teor\'ia econ\'omica de los hogares trata de describir las estructuras del comportamiento de los hogares. Para esto se necesita informaci\'on sobre la estructura demogr\'afica del mismo, la distribuci\'on de recursos, la estructura jer\'arquica y de toma de decisiones, etc. As\'i, se revisan los modelos del comportamiento de hogares colectivos y unitarios, para concluir que estos carecen de entendimiento de la realidad de los hogares, pues se basan en las teor\'ias de elecci\'on del consumidos, sin tomar en cuenta las din\'amicas colectivas.\\
\\
Uno de los primeros modelos que se analizan es la nueva teor\'ia econ\'omica del hogar de Becker que es una modificaci\'on de la teor\'ia de elecci\'on del consumidor, que incluyen el comportamiento colectivo del hogar. En esta teor\'ia se considera que la familia es la unidad de la sociedad, por lo que una teor\'ia econ\'omica que le concierne debe describir la asignaci\'on de recursos dentro de los mismos y los procesos de maximizaci\'on de utilidad. As\'i, se considera al hogar como una unidad de consumo y producci\'on y se basa en tres supuestos principales: que el hogar maximiza el comportamiento, el equilibrio de mercado y las preferencias estables. De este modo, pueden enmarcarse decisiones de fecundidad, por ejemplo, argumentando que tener un ni$\tilde{n}$o adicional conlleva un alto costo de tiempo y econ\'omico , por lo que se disminuye la demanda de familias grandes y reduciendo la fertilidad. Las restricciones a la producci\'on del hogar se deben al tiempo disponible limitado y a los ingresos disponibles, al igual que el consumo. El tiempo se puede convertir en bienes de mercado si gasta m\'as en el tiempo productivo que en el de consumo y un aumento en las ganancias aumenta el costo de oportunidad del tiempo gastado en la categor\'ia de consumo y los precios relativos de los productos b\'asicos cambian, lo que reduce el consumo de aquellos productos que se vuelven m\'as costosos en favor de aquellos cuyos precios no han subido. Tambi\'en, este modelo tiene la hip\'otesis de que cuando un miembro del hogar tiene un aumento en los ingresos, otros miembros deben renunciar a sus carreras productivas para dedicarse al consumo, indicando la sobrevaloraci\'on del trabajo de mercado. Otro de los problemas, es que no incorpora el componente social de los integrantes del hogar, ignorando las ambiciones personales. \\
\\
Otro de los modelos estudiados es el de humano capital humano. En este modelo se argumenta que el costo de oportunidad del trabajo de mercado y los precios no pueden explicar la totalidad del comportamiento humano. El humano capital humano fomenta el desarrollo de relaciones sociales duraderas para crear un sentimiento de pertenencia en el grupo social y no est\'a vinculada a la riqueza del individuo. El concepto de este modelo implica que el bienestar de un individuo y de un hogar no es posible sin la interacci\'on con el otro. As\'i, la adquisici\'on de humano capital humano es un comportamiento colectivo .\\
\\
El modelo colectivo del hogar, tambi\'en conocidos como modelos de toma de decisiones pluralistas, abarcan factores de impacto colectivo dentro del hogar. Estos modelos se enfocan en saber c\'omo las preferencias individuales inducen una elecci\'on colectiva. Aunque estos modelos derivan de manera directa de los neocl\'asicos y consideran las decisiones de un hogar como eficiente si se alcanza el equilibrio de Pareto. No se teoriza sobre la distribuci\'on de recursos dentro del hogar.\\
\\
El mayor problema con estos modelos son la sobresimplificaci\'on de la realidad, dado que la econom\'ia de la esfera dom\'estica son mucho m\'as complejas de lo que los modelos neocl\'asicos pudieran incorporar. Una de las principales cr\'iticas reside en la designaci\'on de un patriarca en el cual es el \'unico tomador de decisiones dentro del hogar, de este modo, otrso grupos como madres solteras, familias uniparentales o multigeneracionales no se adhieren al modelo de decisi\'on. Otra es que en los modelos neocl\'asicos, se asume que el individuo siempre est\'a intentando maximizar su propia utilidad, mientras que en la vida real los lazos afectivos y los sentimientos altruistas juegan un importante papel en la decisi\'on del consumidos, particularmente, se observa en las mujeres que anteponen el bienestar del hogar que del suyo propio.\\
\\
Por lo que las teor\'ias neocl\'asicas toman como supuestos afirmaciones simplistas del comportamiento humano, dado que el bienestar del hogar no depende de manera exclusiva de la aplicaci\'on racional de los principios de la econom\'ia del mercado. Tambi\'en el bienestar depende de cuestiones de seguridad, humanidad y las relaciones sociales. Es por eso que se conlcuye con la necesidad de elaborar teor\'ias econ\'omicas del hogar m\'as realistas, incorporando aspectos humanos y sociales y dejando al mercado como un caso especial.
\bibliographystyle{apalike}
\bibliography{bib_COLMEX}
\end{document}
\documentclass[11pt,spanish,letterpaper]{article}
\usepackage[spanish,activeacute]{babel}
\usepackage{caption}
\usepackage{subcaption}
%\usepackage[latin1]{inputenc}
\usepackage{babel}
\usepackage{picture}
\usepackage{url}
\usepackage{mathrsfs}
\usepackage{amssymb,amsthm,amsmath,latexsym}
\usepackage[round]{natbib}
\usepackage{fancyhdr}
\theoremstyle{plain}
\newtheorem{teo}{Teorema}
\newtheorem{prop}[teo]{Proposicion}
\newtheorem{defi}[teo]{Definicion}
\newtheorem{obs}[teo]{Observacion}
\newtheorem{lem}[teo]{Lema}
\newtheorem{cor}[teo]{Corolario}
\usepackage[pdftex]{color,graphicx}
\newcommand{\sgn}{\mathop{\mathrm{sgn}}}
\setlength{\textwidth}{12.6cm}
\setlength{\textheight}{19cm}
\begin{document}
\begin{flushleft}
Adriana P\'erez-Arciniega Sober\'on
\end{flushleft}
\begin{center}
\textbf{Toward an Ecological Demography}
\end{center}
\cite{low1992toward} argumentan que tanto los dem\'ografos como los bi\'ologos comparten intereses. Principalmente, el inter\'es en el estudio del crecimiento natural de las poblaciones, aunque lo estudien desde enfoques distintos; sin embargo, juntando ambas \'opticas se podr\'ian explorar nuevos enfoques sobre los comportamientos reproductivos en las poblaciones. Podr\'ia plantearse como que los bi\'ologos ven la vida de un solo individuo, mientras que el dem\'ografo estudia a la poblaci\'on agregada; estas diferencias indican que las medidad demogr\'aficas son inadecuadas para resolver problemas ecol\'ogicos y viceversa.\\
\\
Se podr\'ia concluir que la especie humana comparte caracter\'isticas o problemas ecol\'ogicos con otras poblaciones. Uno de los mayores problemas ecol\'ogicos que enfrentan las poblaciones es sobre la asignaci\'on \'optima de esfuerzos, es por esto que el m\'aximo de fecundidad no es el mejor en t\'erminos reproductivos.\\
\\
Una de las disciplinas que trata de incorporar ambos enfoques es la demograf\'ia ecol\'ogica, en la cual se estudia la correlaci\'on que existe entre los rasgos de la poblaci\'on y las condiciones del medio ambiente; se exploran la gen\'etica poblacional y la transmisi\'on cultural de los patrones. Este estudio se vuelve crucial cuando se busca intervenir de manera m\'edica o de planificaci\'on, en el comportamiento de la poblaci\'on. Cuando se quiere hacer predicciones, estos dos enfoques no suelen converger, sobre todo en fecundidad natural, control de la poblaci\'on y la asignaci\'on de recursos existentes o tener ni$\tilde{n}$os adicionales.\\
\\
En lo que se refiere a la fecundidad natural, los dem\'ografos han percibido que la natalidad var\'ia en sociedades aun sin el uso de un control anticonceptivo consistente, por lo que se podr\'ia inferir la existencia de factores biol\'ogicos o sociales que  influyen en la fecundidad, como podr\'ia ser un proceso de adaptaci\'on o una respuesta a las condiciones ecol\'ogicas de la poblaci\'on. En el tema de control de la poblaci\'on, estos dos enfoques divergen en sus predicciones sobre como responder\'a la sociedad ante un cambio en las circunstancias, aunque ambos esperar\'ia ver patrones generales iguales como crecimiento en condiciones de riqueza y estancamiento en condiciones de pobreza; esto no toma en cuenta la individualidad de las poblaciones y la disponibilidad inmediata de los recursos.\\
\\
As\'i, se puede ver que ambas disciplinas se benefician de la mutua colaboraci\'on. Muchos problemas demogr\'aficos se ven enriquecidos al ser estudiados bajo una \'optica biol\'ogica y muchas preguntas sobre el comportamiento de otras especies, puede ser resulto con la ayuda de datos demogr\'aficos humanos. 
\bibliographystyle{apalike}
\bibliography{bib_COLMEX}
\end{document}
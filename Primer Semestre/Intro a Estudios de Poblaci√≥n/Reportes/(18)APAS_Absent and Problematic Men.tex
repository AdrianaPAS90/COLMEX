\documentclass[11pt,spanish,letterpaper]{article}
\usepackage[spanish,activeacute]{babel}
\usepackage{caption}
\usepackage{subcaption}
%\usepackage[latin1]{inputenc}
\usepackage{babel}
\usepackage{picture}
\usepackage{url}
\usepackage{mathrsfs}
\usepackage{amssymb,amsthm,amsmath,latexsym}
\usepackage[round]{natbib}
\usepackage{fancyhdr}
\theoremstyle{plain}
\newtheorem{teo}{Teorema}
\newtheorem{prop}[teo]{Proposicion}
\newtheorem{defi}[teo]{Definicion}
\newtheorem{obs}[teo]{Observacion}
\newtheorem{lem}[teo]{Lema}
\newtheorem{cor}[teo]{Corolario}
\usepackage[pdftex]{color,graphicx}
\newcommand{\sgn}{\mathop{\mathrm{sgn}}}
\setlength{\textwidth}{12.6cm}
\setlength{\textheight}{19cm}
\begin{document}
\begin{flushleft}
Adriana P\'erez-Arciniega Sober\'on
\end{flushleft}
\begin{center}
\textbf{Absent and Problematic Men: Demographic Accounts of Male Reproductive Roles}
\end{center} 
\cite{greene2000absent} discuten la ausencia de los hombres en los estudios sobre reproducci\'on y fecundidad, debido a la preponderancia de las mujeres en los roles de procreaci\'on y a la creencia que los hombres no son indispensables para entender el comportamiento reproductivo. As\'i el art\'iculo se enfoca en describir la raz\'on de esta ausencia, caracterizar la investigaci\'on existente y sugiere futuros enfoques de investigaci\'on concerniendo a los hombres y a la poblaci\'on.\\
\\
Debido a que la demograf\'ia es un campo que, de la manera m\'as simplificada, se dedica a contar sucesos (nacimentos, muertes, migraci\'on, etc) que est\'a influenciado por normas sociales, se considera a los hombres como una unidad econ\'omica y, respecto a la fecundidad, solamente forma parte del proceso de concepci\'on e impidiendo a las mujeres el uso de anticonceptivos, dejando a las mujeres como las responsables totales de la procreaci\'on. Sin embargo, esta visi\'on no toma en cuenta las relaciones de poder dentro de las parejas. Otra raz\'on por la cual no se han incluido a los varones en estudios de fecundidad es puramente metodol\'ogica, debido a que los per\'iodos reproductivos de los hombres no est\'an tan definidos como los de las mujeres; ellas est\'an m\'as en casa, lo que las hace m\'as f\'aciles de entrevistar y, los hijos es m\'as probable que vivan con la madre.\\
\\
Sin embargo, a pesar de las barreras mencionadas el inter\'es del papel masculino en la fecundidad cobra cada vez m\'as inter\'es. Una de las razones del creciente inter\'es se debe a la influencia del feminismo en el \'ambito acad\'emico en la segunda mitad del siglo XX, esto promueve diversificar el papel de la mujer m\'as all\'a del hogar y esto provoca una reevaluaci\'on de todos los integrantes del hogar. Otra raz\'on es el movimiento de salud de la mujer, cambiando el enfoque de planificac\'on familiar a salud reproductiva, haciendo evidente la necesidad de que los hombres se hagan m\'as responsables y consientes de sus decisiones. A su vez, el fracaso de la teor\'ia de la trancisi\'on demogr\'afica por lo que se promueve m\'as investigaciones sobre el comportamiento reproductivo y sobre los variados roles que toman los hombres y las mujeres en la reproducci\'on, de acuerdo al contexto cultural.\\
\\
Aunque las investigaciones de los hombres en la fecundidad han aumentado, estas investigaciones se han enfocado a tratar de resolver los problemas sociales, es decir, que las preguntas de investigaci\'on del rol masculino es tratado como problem\'atico. As\'i, debido a que asumimos que el embarazo y la crianza de los ni$\tilde{n}$os son una preocupaci\'on femenina, no se ha investigado mucho la visi\'on de los hombres respecto al control natal, como anticonceptivos, aborto. Sin embargo, nuevas investigaciones demuestran el inter\'es y aprobaci\'on de los hombres sobre anticonceptivos. Incluso uno de los indicadores de la responsabilidad masculina en la reproducci\'on con el uso de anticonceptivos masculinos, aunque esto no es indicador de el compromiso hacia otros aspectos de la responsabilidad reproductiva.\\
\\
Es com\'un encontrar en la literatura la idea de que los hombres son los que previenen a las mujeres de tomar medidas anticonceptivas, bas\'andose en tres argumentos. El pronatalista, donde se plantea que los hombres quieren m\'as hijos que las mujeres; aunque los datos agregados demuestran que ambos quieren un n\'umero similar de hijos, los desacuerdos surgen a nivel pareja, aunque esto no implica un enfoque pronatalista de parte de los hombres. Los desacuerdos entre pareja, donde es posible que los hombres tengan menos influencia de la que se piensa y el poder de decisi\'on, en el cual se argumenta que a pesar de los desacuerdos y qui\'en se supone que tiene la \'ultima palabra, las mujeres siempre pueden tomar medidas anticonceptivas sin el conocimiento de su pareja.\\
\\
Otro punto que hace interesante el estudio de los hombres en temas de reproducci\'on es que son sexualmente m\'as activos y tienen m\'as parejas, lo cual conlleva m\'as probabilidad de tener hijos; esto tiene una conexi\'on directa con el compromiso a los hijos que efectivamente resulta de sus encuentros. Este compromiso toma forma de trabajo dom\'estico, labores de cuidado y aportaciones financieras; con el aumento en divorcios, se vuelve m\'as complicado estudiar el rol parental masculino debido a que, en primer lugar, los hombres tienen hijos de m\'as de una uni\'on sexual y dado que los enfoques de vestigaci\'on siemore se refieren a los hijos biol\'ogicos no se mira la inversi\'on que hacen muchos hombres a hijos ajenos.\\
\\
En conclusi\'on, el enfoque actual enfocado a la soluci\'on de problemas es limitado porque no considera como el hombre se ve a si mismo y a sus funciones reproductivas. As\'i que se deben entender las motivaciones masculinas en la reproducci\'on y el compromiso con la paternidad, c\'omo esto difiere de las mujeres y c\'omo podemos integrar esta informaci\'on en an\'alisis y pol\'iticas reproductivas.
\bibliographystyle{apalike}
\bibliography{bib_COLMEX}
\end{document}
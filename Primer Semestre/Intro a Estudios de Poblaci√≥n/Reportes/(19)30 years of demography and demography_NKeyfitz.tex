\documentclass[11pt,spanish,letterpaper]{article}
\usepackage[spanish,activeacute]{babel}
\usepackage{caption}
\usepackage{subcaption}
%\usepackage[latin1]{inputenc}
\usepackage{babel}
\usepackage{picture}
\usepackage{url}
\usepackage{mathrsfs}
\usepackage{amssymb,amsthm,amsmath,latexsym}
\usepackage[round]{natbib}
\usepackage{fancyhdr}
\theoremstyle{plain}
\newtheorem{teo}{Teorema}
\newtheorem{prop}[teo]{Proposicion}
\newtheorem{defi}[teo]{Definicion}
\newtheorem{obs}[teo]{Observacion}
\newtheorem{lem}[teo]{Lema}
\newtheorem{cor}[teo]{Corolario}
\usepackage[pdftex]{color,graphicx}
\newcommand{\sgn}{\mathop{\mathrm{sgn}}}
\setlength{\textwidth}{12.6cm}
\setlength{\textheight}{19cm}
\begin{document}
\begin{flushleft}
Adriana P\'erez-Arciniega Sober\'on
\end{flushleft}
\begin{center}
\textbf{Thirty years of Demography and \textit{Demography}}
\end{center} 
El art\'iculo de \cite{keyfitz1993thirty} hace un an\'alisis sobre la revista \textit{Demography} y sobre los temas de investigaci\'on que ha tocado en los treinta a$\tilde{n}$os de su existencia, conforme esta ha ido creciendo. \textit{Demography} se ha vuelto un basti\'on en el campo junto con \textit{Population Studies}, la revista que m\'as se parece en intereses aunque con un enfoque m\'as hist\'orico, \textit{Population Index} uno de los m\'as grandes referentes sobre la bibliograf\'ia actual, \textit{Population}, que est\'a m\'as interesada en monitorear los cambios en la fertilidad y mortalidad europea y \textit{The Population and Development Review},  que se enfoca en los pa\'ises  en  desarrollo,  documenta  la  historia  y  la  prehistoria  de  la  disciplina.\\
\\
La revista \textit{Demography} surgi\'o casi al mismo tiempo que los avances tecnol\'ogicos y la computadora, permitiendo realizar an\'alisis demogr\'aficos m\'as detallados y con una mayor manipulaci\'on de los datos, logrando de este modo, art\'iculos mucho m\'as sostenidos en an\'alisis emp\'irico. Esto logr\'o la popularidad de la revista y su seriedad, logrando que varios de los art\'iculos ah\'i publicados se tomaran como referentes en futuras investigaciones.\\
\\
Lo que se comentaba en el art\'iculo de \cite{poirier1999trente} sobre la compartimentaci\'on del conocimiento tambi\'en se menciona aqu\'i, sobre como muchas disciplinas como la biolog\'ia, ingenieros de operaciones o actuarios estudian la mortalidad y la supervivencia de la misma manera pero no con los mismos m\'etodos. Es por esto que la divulgaci\'on del conocimiento mediante revistas se vuelve de vital importancia para el campo y la comunicaci\'on de informaci\'on; aunque esto tambi\'en de pauta para que sean los editores de dichas revistas los que dicten los temas que merecen investigaci\'on y esto permea hasta el p\'ublico en general.\\
\\
El art\'iculo tambi\'en menciona diferentes l\'ineas de investigaci\'on futuras, como mortalidad, fecundidad, migraci\'on internacional, migraci\'on rural-urbana y poblaci\'on y desarrollo. Los nuevos temas de mortalidad, dado que ya se ha avanzado en el tema de las condiciones m\'edicas y mortalidad temprana, se busca conocer las nuevas esperanzas de vida y el fin de la vida activa. En la fertilidad, se busca conocer como subir la tasa de natalidad en los pa\'ises desarrollados a la tasa de reemplazo generacional de 2.1 hijos por mujer y en los pa\'ises en v\'ias de desarrollo a bajar la tasa de natalidad a 2.1 hijos por mujer; aunado con el problema de la fecundidad, se busca estudiar por qu\'e la instituci\'on de la familia se ha debilitado mientras que el divorcio aumenta. En el tema de migraci\'on internacional y mobilidad rural-urbana, se refiere a un fen\'omeno muy importante de la din\'amica social que se refiere a poca tolerancia a lo ajeno y a la presi\'on que ejercen los inmigrantes para obtener los servicios b\'asicos en su nueo lugar de residencia. Por \'ultimo, el tema de poblaci\'on y desarrollo se refiere a la interacci\'on de los fen\'omenos demogr\'aficos entre s\'i y con el ambiente.\\
\\
Como conclusi\'on, se refiere a la demograf\'ia como un campo de estudio dividido entre dos grupos. Un grupo piensa que el otro utiliza la demograf\'ia como una excusa para hacer matem\'aticas que no se sostendr\'ian en el campo de las matem\'aticas formales y el otro grupo piensa que los que hacen investigaci\'on cualitativa emp\'irica no contribuyen en el hacer cient\'ifico de la disciplina. En este debate interno de la demograf\'ia se debe buscar el punto medio porque al final del d\'ia, nadie hace aportaciones solo, siempre se necesita del saber del otro.
\bibliographystyle{apalike}
\bibliography{bib_COLMEX}
\end{document}